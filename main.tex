%xelatex 或 pdflatex 编译
%导言区
\documentclass[UTF8]{article} %UTF8编码
%preface for the main.tex
\usepackage[a4paper]{geometry} %设置纸张为A4大小
\usepackage{amsmath} %最常用的数学宏包
\allowdisplaybreaks %允许多行公式换页
\usepackage{amssymb} %提供了额外的数学字体和数学符号
\usepackage{mathtools} %提供了额外的数学工具,如dcases环境
\usepackage{lipsum} %英语假文宏包
\usepackage{graphicx} %插入图片宏包
\usepackage{float} %在浮动体中可使用H选项
\usepackage{extarrows} %提供了额外的箭头,如可随文字延长的各种箭头
\usepackage{array} %可增加更多的表格列限定符
\usepackage{dcolumn} %可使用小数点对齐的列
\usepackage{booktabs} %可使用三线表
\usepackage{bm} %可在数学环境中使用斜体加粗的命令
\usepackage[dvipsnames]{xcolor} %扩展版的color宏包

\usepackage{hyperref} %可使用超链接(含链接的跳转和一些命令)
\hypersetup{
	colorlinks=true,
	citecolor=magenta,%设置cite类命令超链接的颜色
	linkcolor=blue,%设置目录、脚注、ref等超链接的颜色
	urlcolor=violet,%设置网页超链接的颜色
}

\usepackage{listings}
% Description: Stata language definition for listings
% Author: WANG Meiting, PhD, Institute for Economic and Social Research, Jinan University
% Email: wangmeiting92@gmail.com
% Created on Sep 15, 2020
%
% Inspired by "https://git.io/JU8tn" and "https://git.io/JU8tG"

\lstdefinelanguage{Stata}{
    % 首要关键字
    morekeywords=[1]{qui,cls,eststo,ivregress,esttab,2sls,regress,reg,summarize,sum,display,di,generate,gen,bysort,use,import,delimited,predict,quietly,probit,margins,test,forvalues,foreach,set,sysuse,graph,local,global,export,twoway,replace,append,clear,if,else,dis,order,mat,matrix,mata,excel,cd,logit,aggregate,array,boolean,break,byte,case,catch,class,colvector,complex,const,continue,default,delegate,delete,do,double,eltypedef,end,enum,explicit,external,float,for,friend,function,goto,inline,int,long,namespace,new,numeric,NULL,operator,orgtypedef,pointer,polymorphic,pragma,private,protected,public,quad,real,return,rowvector,scalar,short,signed,static,strL,string,struct,super,switch,template,this,throw,transmorphic,try,typedef,typename,union,unsigned,using,vector,version,virtual,void,volatile,while,in,define,rown,rownames,coln,colnames,list,corr,ttest,tokenize},
    % 函数关键字
    morekeywords=[2]{bofd,Cdhms,Chms,Clock,clock,Cmdyhms,Cofc,cofC,Cofd,cofd,daily,date,day,dhms,dofb,dofC,dofc,dofh,dofm,dofq,dofw,dofy,dow,doy,halfyear,halfyearly,hh,hhC,hms,hofd,hours,mdy,mdyhms,minutes,mm,mmC,mofd,month,monthly,msofhours,msofminutes,msofseconds,qofd,quarter,quarterly,seconds,ss,ssC,tC,tc,td,th,tm,tq,tw,week,weekly,wofd,year,yearly,yh,ym,yofd,yq,yw,abs,ceil,cloglog,comb,digamma,exp,expm1,floor,int,invcloglog,invlogit,ln,ln1m,ln,ln1p,ln,lnfactorial,lngamma,log,log10,log1m,log1p,max,min,mod,reldif,round,sign,sqrt,trigamma,trunc,cholesky,coleqnumb,colnfreeparms,colnumb,colsof,det,diag,diag0cnt,el,get,hadamard,I,inv,invsym,issymmetric,J,matmissing,matuniform,mreldif,nullmat,roweqnumb,rownfreeparms,rownumb,rowsof,sweep,trace,vec,vecdiag,autocode,byteorder,c,_caller,chop,abs,clip,cond,e,fileexists,fileread,filereaderror,filewrite,float,fmtwidth,has_eprop,inlist,inrange,irecode,maxbyte,maxdouble,maxfloat,maxint,maxlong,mi,minbyte,mindouble,minfloat,minint,minlong,missing,r,recode,replay,return,s,scalar,smallestdouble,rbeta,rbinomial,rcauchy,rchi2,rexponential,rgamma,rhypergeometric,rigaussian,rlaplace,rlogistic,rnbinomial,rnormal,rpoisson,rt,runiform,runiformint,rweibull,rweibullph,tin,twithin,betaden,binomial,binomialp,binomialtail,binormal,cauchy,cauchyden,cauchytail,chi2,chi2den,chi2tail,dgammapda,dgammapdada,dgammapdadx,dgammapdx,dgammapdxdx,dunnettprob,exponential,exponentialden,exponentialtail,F,Fden,Ftail,gammaden,gammap,gammaptail,hypergeometric,hypergeometricp,ibeta,ibetatail,igaussian,igaussianden,igaussiantail,invbinomial,invbinomialtail,invcauchy,invcauchytail,invchi2,invchi2tail,invdunnettprob,invexponential,invexponentialtail,invF,invFtail,invgammap,invgammaptail,invibeta,invibetatail,invigaussian,invigaussiantail,invlaplace,invlaplacetail,invlogistic,invlogistictail,invnbinomial,invnbinomialtail,invnchi2,invnF,invnFtail,invnibeta,invnormal,invnt,invnttail,invpoisson,invpoissontail,invt,invttail,invtukeyprob,invweibull,invweibullph,invweibullphtail,invweibulltail,laplace,laplaceden,laplacetail,lncauchyden,lnigammaden,lnigaussianden,lniwishartden,lnlaplaceden,lnmvnormalden,lnnormal,lnnormalden,lnwishartden,logistic,logisticden,logistictail,nbetaden,nbinomial,nbinomialp,nbinomialtail,nchi2,nchi2den,nchi2tail,nF,nFden,nFtail,nibeta,normal,normalden,npnchi2,npnF,npnt,nt,ntden,nttail,poisson,poissonp,poissontail,t,tden,ttail,tukeyprob,weibull,weibullden,weibullph,weibullphden,weibullphtail,weibulltail,abbrev,char,collatorlocale,collatorversion,indexnot,plural,plural,real,regexm,regexr,regexs,soundex,soundex_nara,strcat,strdup,string,strofreal,string,strofreal,stritrim,strlen,strlower,strltrim,strmatch,strofreal,strofreal,strpos,strproper,strreverse,strrpos,strrtrim,strtoname,strtrim,strupper,subinstr,subinword,substr,tobytes,uchar,udstrlen,udsubstr,uisdigit,uisletter,ustrcompare,ustrcompareex,ustrfix,ustrfrom,ustrinvalidcnt,ustrleft,ustrlen,ustrlower,ustrltrim,ustrnormalize,ustrpos,ustrregexm,ustrregexra,ustrregexrf,ustrregexs,ustrreverse,ustrright,ustrrpos,ustrrtrim,ustrsortkey,ustrsortkeyex,ustrtitle,ustrto,ustrtohex,ustrtoname,ustrtrim,ustrunescape,ustrupper,ustrword,ustrwordcount,usubinstr,usubstr,word,wordbreaklocale,worcount,acos,acosh,asin,asinh,atan,atanh,cos,cosh,sin,sinh,tan,tanh,},
    % 注释
    morecomment=[l][\itshape\color{gray}]{//},
    morecomment=[s][\color{gray}]{/*}{*/},
    morecomment=[f][\itshape\color{gray}][0]{*}, %当*在首列时,为注释
    morecomment=[f][\itshape\color{gray}][1]{*}, %当*在第二列时,为注释
    morecomment=[f][\itshape\color{gray}][2]{*}, %当*在第三列时,为注释
    % 字符串
    morestring=[s]{`}{'},
    morestring=[s]{"}{"},
    morestring=[s]{"`}{'"},
    morestring=[s]{`"`}{'"'},
} %导入Stata的listings配置文件
\lstset{ % Stata 语言全局设置
	language=Stata,
	basicstyle={\ttfamily},
% 	commentstyle={\color{gray}}, %设置注释的文字格式
	keywordstyle={[1]\color{Magenta}}, %设置首要关键字的文字格式
	keywordstyle={[2]\color{blue}}, %设置函数关键字的文字格式
	stringstyle={\color{Purple}}, %设置字符串的文字格式
	emph={}, %使前面 LaTeX 语言所设定的 emph 失效
	breaklines=true,
	tabsize=4,
	frame=trBL,
	frameround=fttt,
	flexiblecolumns=true,
}

\usepackage[amsmath,thmmarks]{ntheorem} %定理类环境宏包,如果前面使用amsmath宏包,则需加上amsmath宏包选项以避免出现未知问题,若需在定理环境末尾加上特定符号(如证毕符号),则需使用thmmarks宏包选项以使用\theoremsymbol{}命令。
{
	\theoremstyle{plain}
	\setlength{\theoremindent}{0em}
	\newtheorem{definition}{Definition}
}
{
	\theoremstyle{plain}
	\setlength{\theoremindent}{0em}
	\newtheorem{lemma}[definition]{Lemma}
}
{
	\theoremstyle{plain}
	\setlength{\theoremindent}{0em}
	\newtheorem{theorem}[definition]{Theorem}
}
{
	\theoremstyle{plain}
	\setlength{\theoremindent}{0em}
	\newtheorem{corollary}[definition]{Corollary}
}
{
	\theoremstyle{nonumberplain}
	\setlength{\theoremindent}{0em}
	\theorembodyfont{\normalfont}
	\theoremsymbol{$\blacksquare$} %在证明环境末尾加上一个证毕符号
	\newtheorem{proof}{Proof}
}

%定义新命令与新运算符
\DeclareMathOperator{\diff}{d\!}
\newcommand{\var}{\mathrm{Var}}
\newcommand{\avar}{\mathrm{Avar}}
\newcommand{\cov}{\mathrm{Cov}}
\newcommand{\E}{\mathrm{E}}

%重定义列表
\renewcommand{\labelenumii}{\theenumii.}

%标题页设置
\title{
	Problem sets for Advanced Econometrics III
}
\author{Meiting Wang\thanks{Meiting Wang, Email: wangmeiting92@gmail.com}}
\date{\today}




%正文区
\begin{document}

\maketitle
\tableofcontents
\pagenumbering{Roman}


\clearpage
\pagenumbering{arabic}
\section*{Chapter 5}
\addcontentsline{toc}{section}{Chapter 5}
\begin{enumerate}
    %problem boundary--------------------------------------------------------------------
    \item[5.1] In this problem you are to establish the algebraic equivalence between 2SLS and OLS estimation of an equation containing an additional regressor. Although the result is completely general, for simplicity consider a model with a single (suspected) endogenous variable: 
    \begin{gather*}
        y_{1}=\mathbf{z}_{1} \boldsymbol{\delta}_{1}+\alpha_{1} y_{2}+u_{1} \\
        y_{2}=\mathbf{z} \pi_{2}+v_{2}
    \end{gather*}
    For notational clarity, we use $y_{2}$ as the suspected endogenous variable and $\mathbf{z}$ as the vector of all exogenous variables. The second equation is the reduced form for $y_{2}$. Assume that $z$ has at least one more element than $\mathbf{z}_{1}$. We know that one estimator of $\left(\delta_{1}, \alpha_{1}\right)$ is the $2 \mathrm{SLS}$ estimator using instruments $\mathbf{z}$. Consider an alternative estimator of $\left(\delta_{1}, \alpha_{1}\right):$ (a) estimate the reduced form by OLS, and save the residuals $\hat{v}_{2} ;(\mathrm{b})$ estimate the following equation by OLS:
    \begin{gather}
    y_{1}=\mathbf{z}_{1} \delta_{1}+\alpha_{1} y_{2}+\rho_{1} \hat{v}_{2}+error \tag{5.52} \label{eq:5.1-1}
    \end{gather}
    Show that the OLS estimates of $\boldsymbol{\delta}_{1}$ and $\alpha_{1}$ from this regression are identical to the 2SLS estimators. (Hint: Use the partitioned regression algebra of OLS. In particular, if $\hat{y}=\mathbf{x}_{1} \hat{\boldsymbol{\beta}}_{1}+\mathbf{x}_{2} \hat{\boldsymbol{\beta}}_{2}$ is an OLS regression, $\hat{\boldsymbol{\beta}}_{1}$ can be obtained by first regressing $\mathbf{x}_{1}$ on $\mathbf{x}_{2},$ getting the residuals, say $\ddot{\mathbf{x}}_{1},$ and then regressing $y$ on $\ddot{\mathbf{x}}_{1} ;$ see, for example, Davidson and MacKinnon (1993, Section 1.4). You must also use the fact that $\mathbf{z}_{1}$ and $\hat{v}_{2}$ are orthogonal in the sample.)(\textcolor{red}{see p127})
    
    \textbf{Answer:} To obtain the OLS estimators of the equation \eqref{eq:5.1-1}, as the hint shows, we can do the following alternative things: 
    \begin{itemize}
        \item Get the residuals of the regression $\mathbf{z}_1$ on $\hat{v}_2$. Due to the orthogonality between $\mathbf{z}_1$ and $\hat{v}_2$, the residual will be $\mathbf{z}_1 - \mathbf{cons}$.
        \item Get the residuals of the regression $y_2$ on $\hat{v}_2$. From the regression $y_2$ on $\mathbf{z}$, we can write $y_2 = \hat{y}_2 + \hat{v}_2$, where the $\hat{y}_2$ are the fitted values and the $\hat{v}_2$ are the residuals. It is easy to know that $\hat{y}_2$ is orthogonal with $\hat{v}_2$ from the nature of OLS, so the residuals from the regression $y_2$ on $\hat{v}_2$ will be $\hat{y}_2 - cons$.
        \item Do regression $y_1$ on the residuals obtained above.
    \end{itemize}
    According to the alternative steps to get the OLS estimators of the equation \eqref{eq:5.1-1}, we know that the OLS estimators are identical to the 2SLS estimators showed in the question.
    
    
    %problem boundary--------------------------------------------------------------------
    \item[5.7] Consider model (5.45) where $v$ has zero mean and is uncorrelated with $x_{1}, \cdots, x_{K}$ and $q .$ The unobservable $q$ is thought to be correlated with at least some of the $x_{j} .$ Assume without loss of generality that $\mathrm{E}(q)=0$. You have a single indicator of $q,$ written as $q_{1}=\delta_{1} q+a_{1}, \delta_{1} \neq 0,$ where $a_{1}$ has zero mean and is uncorrelated with each of $x_{j}, q,$ and $v .$ In addition, $z_{1}, z_{2}, \cdots, z_{M}$ is a set of variables that are (1) redundant in the structural equation (5.45) and (2) uncorrelated with $a_{1}$. (\textcolor{red}{see p112-114})
    \begin{enumerate}
        \item Suggest an IV method for consistently estimating the $\beta_{j} .$ Be sure to discuss what is needed for identification.
        
        \textbf{Answer:} Recall the equation \eqref{eq:5.7-1}
        \begin{gather}
            y=\beta_{0}+\beta_{1} x_{1}+\cdots+\beta_{K} x_{K}+\gamma q+v \tag{5.45} \label{eq:5.7-1}
        \end{gather}
        Plugging $q_{1}=\delta_{1} q+a_{1}$ into the equation above, we have
        \begin{gather}
            y=\beta_{0}+\beta_{1} x_{1}+\cdots+\beta_{K} x_{K}+\frac{\gamma}{\delta_1}q_1 + \left( v-\frac{\gamma}{\delta_1}a_1 \right) \label{eq:5.7-2}
        \end{gather}
        In order to consistently estimate the $\beta_j(j = 0,1,2,\cdots,K)$ by an IV method in equation \eqref{eq:5.7-2}, in addition to the conditions in the question, the following condition is also required: For IV $\mathbf{z}=(1,x_1,\cdots,x_K,z_1,\cdots,z_M)$ in the reduced-form equation
        \begin{gather}
            q_{1}=\pi_{0}+\pi_{1} x_{1}+\ldots+\pi_{K} x_{K}+\theta_{1} z_{1}+\theta_2 z_2 + \cdots+\theta_{M} z_{M}+r_{1} \label{eq:5.7-3}
        \end{gather}
        at least one of $\theta_1, \theta_2, \cdots, \theta_M$ should be different from zero.
        
        \item If equation (5.45) is a $\log(wage)$ equation, $q$ is ability, $q_{1}$ is $I Q$ or some other test score, and $z_{1}, \ldots, z_{M}$ are family background variables, such as parents' education and number of siblings, describe the economic assumptions needed for consistency of the the IV procedure in part a.
        
        \textbf{Answer:} The economic meanings of the conditions in part a are
        \begin{itemize}
            \item The family background variables are redundant in the $\log(wage)$ equation after ability and factors have been controlled for. That is, the family background variables may affect ability but should have no partial effect on $\log(wage)$ after ability and other factors have been accounted for.
            \item The family background variables should have partial effect on indicator($q_1$), after the $x_j(j=1,2,\cdots,K)$ have been controled in equation \eqref{eq:5.7-3}.
        \end{itemize}
        
        \item Carry out this procedure using the data in NLS80.RAW. Include among the explanatory variables \textit{exper, tenure, educ, married, south, urban,} and \textit{black}. First use $I Q$ as $q_{1}$ and then $K W W .$ Include in the $z_{h}$ the variables \textit{meduc, feduc, and sibs}. Discuss the results.
        
        \textbf{Answer:} Using the Stata code below, Firstly, we test the rank conditions when $q_1$ are $IQ$ and $KWW$, and the $F$ statistic for joint significance of \textit{meduc, feduc} and \textit{sibs} have p-values below 0.002, so the rank conditions have been met in a statistical sense. Secondly, we obtain the regression results of OLS and 2SLS in Table \ref{tab:5.7-c}. We can see that the return to education is small and insignificant whether $IQ$ or $KWW$ is used as the indicator(2SLS-iq and 2SLS-kww). This could be because family background variables don't satisfy the redundancy condition or they might be correlated with $a_1$.
        \begin{lstlisting}
use nls80.dta, clear

local z1 "exper tenure educ married south urban black"
local z2 "meduc feduc sibs"

*test rank condition
qui reg iq `z1' `z2', vce(robust)
test `zh'
qui reg kww `z1' `z2', vce(robust)
test `zh'

*output the regression result
eststo clear
eststo: reg lwage `z1' iq, vce(robust)
eststo: reg lwage `z1' kww, vce(robust)
eststo: ivregress 2sls lwage `z1' (iq=`z2'), vce(robust)
eststo: ivregress 2sls lwage `z1' (kww=`z2'), vce(robust) first
esttab *, b(3) se(3) star(* 0.10 ** 0.05 *** 0.01) r2 obslast compress nogaps mti("OLS" "OLS" "2SLS-iq" "2SLS-kww") order(iq kww educ)
\end{lstlisting}
        \begin{table}[H]\centering
\def\sym#1{\ifmmode^{#1}\else\(^{#1}\)\fi}
\caption{Regression results}\vspace{0.2em}\label{tab:5.7-c}
\begin{tabular}{l*{4}{D{.}{.}{-1}}}
\toprule
          &\multicolumn{1}{c}{\hspace{1.3em}(1)}&\multicolumn{1}{c}{\hspace{1.3em}(2)}&\multicolumn{1}{c}{\hspace{1.3em}(3)}&\multicolumn{1}{c}{\hspace{1.3em}(4)}\\
          &\multicolumn{1}{c}{\hspace{1.3em}OLS}&\multicolumn{1}{c}{\hspace{1.3em}OLS}&\multicolumn{1}{c}{\hspace{1.3em}2SLS-iq}&\multicolumn{1}{c}{\hspace{1.3em}2SLS-kww}\\
\midrule
iq        &    0.004\sym{***}&                  &    0.015\sym{**} &                  \\
          &  (0.001)         &                  &  (0.007)         &                  \\
kww       &                  &    0.005\sym{**} &                  &    0.025\sym{*}  \\
          &                  &  (0.002)         &                  &  (0.014)         \\
educ      &    0.054\sym{***}&    0.058\sym{***}&    0.016         &    0.026         \\
          &  (0.007)         &  (0.007)         &  (0.025)         &  (0.024)         \\
exper     &    0.014\sym{***}&    0.012\sym{***}&    0.016\sym{***}&    0.007         \\
          &  (0.003)         &  (0.003)         &  (0.004)         &  (0.006)         \\
tenure    &    0.011\sym{***}&    0.011\sym{***}&    0.008\sym{**} &    0.005         \\
          &  (0.003)         &  (0.003)         &  (0.003)         &  (0.004)         \\
married   &    0.200\sym{***}&    0.189\sym{***}&    0.190\sym{***}&    0.161\sym{***}\\
          &  (0.039)         &  (0.039)         &  (0.047)         &  (0.053)         \\
south     &   -0.080\sym{***}&   -0.092\sym{***}&   -0.048         &   -0.092\sym{***}\\
          &  (0.028)         &  (0.027)         &  (0.040)         &  (0.033)         \\
urban     &    0.182\sym{***}&    0.176\sym{***}&    0.187\sym{***}&    0.148\sym{***}\\
          &  (0.027)         &  (0.027)         &  (0.032)         &  (0.040)         \\
black     &   -0.143\sym{***}&   -0.164\sym{***}&    0.040         &   -0.042         \\
          &  (0.038)         &  (0.039)         &  (0.105)         &  (0.082)         \\
\_cons    &    5.176\sym{***}&    5.359\sym{***}&    4.472\sym{***}&    5.218\sym{***}\\
          &  (0.121)         &  (0.113)         &  (0.441)         &  (0.157)         \\
\midrule
\(R^{2}\) &    0.263         &    0.259         &    0.155         &    0.156         \\
\(N\)     &      \multicolumn{1}{c}{\hspace{1.4em}935}         &      \multicolumn{1}{c}{\hspace{1.4em}935}         &      \multicolumn{1}{c}{\hspace{1.4em}722}         &      \multicolumn{1}{c}{\hspace{1.4em}722}         \\
\bottomrule
\multicolumn{5}{l}{\footnotesize Robust standard errors in parentheses}\\
\multicolumn{5}{l}{\footnotesize \sym{*} \(p<0.10\), \sym{**} \(p<0.05\), \sym{***} \(p<0.01\)}\\
\end{tabular}
\end{table}
    \end{enumerate}
    
    
    %problem boundary--------------------------------------------------------------------
    \item[5.8] Consider a model with unobserved heterogeneity $(q)$ and measurement error in an explanatory variable:
    \[ y=\beta_{0}+\beta_{1} x_{1}+\cdots+\beta_{K} x_{K}^{*}+q+v \]
    where $e_{K}=x_{K}-x_{K}^{*}$ is the measurement error and we set the coefficient on $q$ equal to one without loss of generality. The variable $q$ might be correlated with any of the explanatory variables, but an indicator, $q_{1}=\delta_{0}+\delta_{1} q+a_{1},$ is available. The measurement error $e_{K}$ might be correlated with the observed measure, $x_{K} .$ In addition to $q_{1},$ you also have variables $z_{1}, z_{2}, \ldots, z_{M}, M \geq 2,$ that are uncorrelated with $v, a_{1},$ and $e_{K}$. (\textcolor{red}{see p112-114})
    \begin{enumerate}
        \item Suggest an IV procedure for consistently estimating the $\beta_{j} .$ Why is $M \geq 2$ required? (Hint: Plug in $q_{1}$ for $q$ and $x_{K}$ for $x_{K}^{*}$, and go from there.)
        
        \textbf{Answer:} Plugging in $q_1$ for $q$ and $x_K$ for $x_K^*$ in the following equation: 
        \[ y=\beta_{0}+\beta_{1} x_{1}+\cdots+\beta_{K} x_{K}^{*}+q+v \]
        We have
        \[ y=\left(\beta_{0}-\frac{\delta_0}{\delta_1}\right) + \beta_{1} x_{1}+\cdots+\beta_{K} x_{K} + \frac{1}{\delta_1}q_1+\left(v-\beta_K e_K-\frac{1}{\delta_1}a_1\right) \]
        From the equation above, it is easy to know that $x_K$ and $q_1$ are the potential endogenous variables because $x_K$ and $q_1$ might be correlated with composite error $v-\beta_K e_K-\frac{1}{\delta_1}a_1$, so $M\geq 2$ is required.
        
        \item Apply this method to the model estimated in Example $5.5,$ where actual education, say educ*, plays the role of $x_{K}^{*}$. Use $I Q$ as the indicator of $q=$ \textit{ability}, and $K W W,$ \textit{meduc, feduc,} and \textit{sibs} as the elements of $\mathbf{z}$.
        
        \textbf{Answer:} Using the Stata code below, Firstly, we test the rank conditions for $x_K$ and $iq$, and the both $F$ statistic for joint significance of \textit{kww, meduc, feduc} and \textit{sibs} have p-values almost close to 0, so the rank conditions have been met in a statistical sense. Secondly, we obtain the regression results in table \ref{tab:5.8-b}. we can see that the estimated return to education is very large but not statistically significant and the coefficient of $iq$ is rarely negative but not statistically different from zero. It is seemingly that omitted $ability$ is less of problem than education measurement error in the standard $\log(wage)$ model estimated by OLS. However, insignificant coefficient for $educ$ make the evidences above not very convincing.
        \begin{lstlisting}
use nls80.dta, clear

local z1 "exper tenure married south urban black"
local z2 "kww meduc feduc sibs"

*test rank condition
qui reg educ `z1' `z2', vce(robust)
test `z2'
qui reg iq `z1' `z2', vce(robust)
test `z2'

*output the regression result
eststo clear
eststo: reg lwage `z1' educ iq, vce(robust)
eststo: ivregress 2sls lwage `z1' (educ iq=`z2'), vce(robust)
esttab *, b(3) se(3) star(* 0.10 ** 0.05 *** 0.01) r2 obslast compress nogaps mti("OLS" "2SLS") order(educ iq)
\end{lstlisting}
        \begin{table}[H]\centering
\def\sym#1{\ifmmode^{#1}\else\(^{#1}\)\fi}
\caption{Regression results}\vspace{0.2em}\label{tab:5.8-b}
\begin{tabular}{l*{2}{D{.}{.}{-1}}}
\toprule
          &\multicolumn{1}{c}{\hspace{1.3em}(1)}&\multicolumn{1}{c}{\hspace{1.3em}(2)}\\
          &\multicolumn{1}{c}{\hspace{1.3em}OLS}&\multicolumn{1}{c}{\hspace{1.3em}2SLS}\\
\midrule
educ      &    0.054\sym{***}&    0.165         \\
          &  (0.007)         &  (0.111)         \\
iq        &    0.004\sym{***}&   -0.010         \\
          &  (0.001)         &  (0.020)         \\
exper     &    0.014\sym{***}&    0.031\sym{***}\\
          &  (0.003)         &  (0.012)         \\
tenure    &    0.011\sym{***}&    0.007\sym{**} \\
          &  (0.003)         &  (0.003)         \\
married   &    0.200\sym{***}&    0.213\sym{***}\\
          &  (0.039)         &  (0.056)         \\
south     &   -0.080\sym{***}&   -0.094\sym{*}  \\
          &  (0.028)         &  (0.051)         \\
urban     &    0.182\sym{***}&    0.168\sym{***}\\
          &  (0.027)         &  (0.037)         \\
black     &   -0.143\sym{***}&   -0.235         \\
          &  (0.038)         &  (0.224)         \\
\_cons    &    5.176\sym{***}&    4.933\sym{***}\\
          &  (0.121)         &  (0.472)         \\
\midrule
\(R^{2}\) &    0.263         &       \multicolumn{1}{c}{\hspace{1.4em}-}        \\
\(N\)     &      \multicolumn{1}{c}{\hspace{1.4em}935}         &      \multicolumn{1}{c}{\hspace{1.4em}722}         \\
\bottomrule
\multicolumn{3}{l}{\footnotesize Robust standard errors in parentheses}\\
\multicolumn{3}{l}{\footnotesize \sym{*} \(p<0.10\), \sym{**} \(p<0.05\), \sym{***} \(p<0.01\)}\\
\end{tabular}
\end{table}
    \end{enumerate}
    
    
    %problem boundary--------------------------------------------------------------------
    \item[5.10] Consider IV estimation of the simple linear model with a single, possibly endogenous, explanatory variable, and a single instrument: (\textcolor{red}{see p59, p101})
    \begin{gather*}
        y=\beta_{0}+\beta_{1} x+u, \mathrm{E}(u)=0,  \operatorname{Cov}(z, u)=0,  \operatorname{Cov}(z, x) \neq 0,  \mathrm{E}\left(u^{2} \mid z\right)=\sigma^{2}
    \end{gather*}
    \begin{enumerate}
        \item Under the preceding (standard) assumptions, show that $\avar\sqrt{N}\left(\hat{\beta}_{1}-\beta_{1}\right)$ can be expressed as $\sigma^{2} /\left(\rho_{z x}^{2} \sigma_{x}^{2}\right),$ where $\sigma_{x}^{2}=\operatorname{Var}(x)$ and $\rho_{z x}=\operatorname{Corr}(z, x) .$ Compare this result with the asymptotic variance of the OLS estimator under Assumptions OLS.1-OLS.3.
        
        \textbf{Answer:} Recall that under Assumptions 2SLS.1-2SLS.3,
        \[ \avar\sqrt{N}\left(\hat{\bm{\beta}}-\bm{\beta}\right) = \sigma^{2}\left(\left[\mathrm{E}\left(\mathbf{x}^{\prime} \mathbf{z}\right)\right]\left[\mathrm{E}\left(\mathbf{z}^{\prime} \mathbf{z}\right)\right]^{-1} \left[\mathrm{E}\left(\mathbf{z}^{\prime} \mathbf{x}\right)\right]\right)^{-1} \]
        It is easy to know that
        \[ \mathrm{E}\left(\mathbf{x}^{\prime} \mathbf{z}\right) = \begin{bmatrix}
            1 & \mathrm{E}z \\
            \mathrm{E}x & \mathrm{E}xz
        \end{bmatrix},
        \left[\mathrm{E}\left(\mathbf{z}^{\prime} \mathbf{z}\right)\right]^{-1} = \frac{1}{\var(z)} \begin{bmatrix}
            \mathrm{E}z^2 & -\mathrm{E}z \\
            -\mathrm{E}z & 1
        \end{bmatrix},
        \mathrm{E}\left(\mathbf{z}^{\prime} \mathbf{x}\right) = \begin{bmatrix}
            1 & \mathrm{E}x \\
            \mathrm{E}z & \mathrm{E}xz
        \end{bmatrix} \]
        Combine the equations above: 
        \[ \left[\mathrm{E}\left(\mathbf{x}^{\prime} \mathbf{z}\right)\right]\left[\mathrm{E}\left(\mathbf{z}^{\prime} \mathbf{z}\right)\right]^{-1} \left[\mathrm{E}\left(\mathbf{z}^{\prime} \mathbf{x}\right)\right] = \frac{1}{\var(z)} \begin{bmatrix}
            \var(z) & \mathrm{E}x \var(z) \\
            \mathrm{E}x \var(z) & \left(\mathrm{E}x\right)^2 \mathrm{E}z^2-2\mathrm{E}x\mathrm{E}z\mathrm{E}xz+\left(\mathrm{E}xz\right)^2
        \end{bmatrix}\]
        The asymptotic variance of the IV estimators will be
        \[\avar\sqrt{N}\left(\hat{\bm{\beta}}-\bm{\beta}\right) = \frac{\sigma^2}{\left[\cov(x,z)\right]^2} \begin{bmatrix}
            \left(\mathrm{E}x\right)^2 \mathrm{E}z^2-2\mathrm{E}x\mathrm{E}z\mathrm{E}xz+\left(\mathrm{E}xz\right)^2 & -\mathrm{E}x \var(z) \\
            -\mathrm{E}x \var(z) & \var(z)
        \end{bmatrix}\]
        Then
        \[\avar\sqrt{N}\left(\hat{\beta}_{1}-\beta_{1}\right) = \frac{\sigma^2\var(z)}{\left[\cov(x,z)\right]^2} = \frac{\sigma^2\sigma_z^2}{\left(\rho_{xz}\sigma_x\sigma_z\right)^2} = \frac{\sigma^2}{\rho_{xz}^2\sigma_x^2}\]
        Under Assumptions OLS.1-OLS.3, we know that the asymptotic variance for the OLS estimator is $\sigma^2/\sigma^2_x$, so the difference between the OLS and IV estimator in asymptotic variance is whether $\rho_{xz}^2$ appears in the denominator.
        
        \item Comment on how each factor affects the asymptotic variance of the IV estimator. What happens as $\rho_{z x} \rightarrow 0$ ?
        
        \textbf{Answer:} The less error variance $\sigma^2$, the more variance in $x$ and a larger correlation between $x$ and $z$ will lead to a small asymptotic variance of the IV estimator. When $\rho_{z x} \rightarrow 0$, the asymptotic variance will increases without bound. This illustrates why an instrument that is only weakly correlated with $x$ can lead to very imprecise IV estimators.
    \end{enumerate}
    
    
    %problem boundary--------------------------------------------------------------------
    \item[5.11] A model with a single endogenous explanatory variable can be written as
    \[ y_{1}=\mathbf{z}_{1} \boldsymbol{\delta}_{1}+\alpha_{1} y_{2}+u_{1}, \mathrm{E}\left(\mathbf{z}^{\prime} u_{1}\right)=\mathbf{0} \]
    where $\mathbf{z}=\left(\mathbf{z}_{1}, \mathbf{z}_{2}\right) .$ Consider the following two-step method, intended to mimic $2 \mathrm{SLS}$: (1) Regress $y_{2}$ on $\mathbf{z}_{2},$ and obtain fitted values, $\tilde{y}_{2}$(That is, $\mathbf{z}_{1}$ is omitted from the firststage regression.); (2) Regress $y_{1}$ on $\mathbf{z}_{1}, \tilde{y}_{2}$ to obtain $\tilde{\boldsymbol{\delta}}_{1}$ and $\tilde{\alpha}_{1} .$ Show that $\tilde{\boldsymbol{\delta}}_{1}$ and $\tilde{\alpha}_{1}$ are generally inconsistent. When would $\tilde{\boldsymbol{\delta}}_{1}$ and $\tilde{\alpha}_{1}$ be consistent? (Hint: Let $y_{2}^{0}$ be the population linear projection of $y_{2}$ on $\mathbf{z}_{2},$ and let $a_{2}$ be the projection error: $y_{2}=\mathbf{z}_{2} \lambda_{2}+a_{2},$ $\mathrm{E}\left(\mathbf{z}_{2}^{\prime} a_{2}\right)=\mathbf{0} .$ For simplicity, pretend that $\lambda_{2}$ is known rather than estimated; that is, assume that $\tilde{y}_{2}$ is actually $y_{2}^{0}$. Then, write
    \[ y_{1}=\mathbf{z}_{1} \boldsymbol{\delta}_{1}+\alpha_{1} y_{2}^{0}+\alpha_{1} a_{2}+u_{1} \]
    and check whether the composite error $\alpha_{1} a_{2}+u_{1}$ is uncorrelated with the explanatory variables.) (\textcolor{red}{see p96-98})
    
    \textbf{Answer:} Plug $y_2=y_2^0+a_2$ into $y_1$,
    \begin{gather}
        y_{1}=\mathbf{z}_{1} \boldsymbol{\delta}_{1}+\alpha_{1} y_{2}^0+\left(\alpha_1 a_2 + u_{1}\right) = \mathbf{z}_{1} \boldsymbol{\delta}_{1}+\alpha_{1} y_{2}^0+v_1 \label{eq:5.11-1}
    \end{gather}
    where $v_1 = \alpha_1 a_2 + u_{1}$. Due to $\mathrm{E}(\mathbf{z}_2^\prime u_1) = \mathrm{E}(\mathbf{z}_2^\prime a_2) = \mathbf{0}$, then $\mathrm{E}(\mathbf{z}_2^\prime v_1) = \mathbf{0} \implies \mathrm{E}(y_2^0 v_1) = 0$. For $\mathbf{z}_1$, although $\mathrm{E}(\mathbf{z}_1^\prime u_1) = \mathbf{0}$, in general, $\mathrm{E}(\mathbf{z}_1^\prime a_2) \neq \mathbf{0}$ because $\mathbf{z}_1$ was not included in the linear projection for $y_2$. That implies $\mathrm{E}(\mathbf{z}_1^\prime v_1) \neq \mathbf{0}$, universally making the OLS estimators($\tilde{\bm{\delta}}_1$ and $\tilde{\alpha_1}$) in step (2) all inconsistent. On the contrary, only in the case of $\mathrm{E}(\mathbf{z}_1^\prime a_2) = \mathbf{0}$, $\tilde{\boldsymbol{\delta}}_{1}$ and $\tilde{\alpha}_{1}$ will be consistent.
\end{enumerate}


\clearpage
\section*{Chapter 6}
\addcontentsline{toc}{section}{Chapter 6}
\begin{enumerate}
    %problem boundary--------------------------------------------------------------------
    \item[6.2] In Problem 5.8b, test the null hypothesis that $educ$ and $IQ$ are exogenous in the equation estimated by 2SLS. (\textcolor{red}{see p126-127})
    
    \textbf{Answer:} Let $\mathbf{z}=(\mathbf{z}_1,\mathbf{z}_2)$, where $\mathbf{z}_1$ are the original exogenous variables and $\mathbf{z}_2$ are the IV variables. We first obtain the residuals($\hat{v}_1$ and $\hat{v}_2$) from the reduced form regressions below:
    \begin{align*}
        educ = \mathbf{z}\bm{\pi}_1 + v_1 \\ 
        IQ = \mathbf{z}\bm{\pi}_2 + v_2 \\ 
    \end{align*}
    Then, do the regression below and test whether the coefficients $\rho_1$ and $\rho_2$ are zero.
    \[ lwage = \mathbf{z}_1\bm{\delta}_1 + \alpha_1 educ + \alpha_2 IQ + \rho_1\hat{v}_1 + \rho_2\hat{v}_2 + error \]
    From the test result below, we find a strong evidence for the endogeneity of at least one of $educ$ and $IQ$. 
    \begin{lstlisting}
use nls80.dta, clear

local z1 "exper tenure married south urban black"
local z2 "kww meduc feduc sibs"

qui reg educ `z1' `z2'
predict v1hat, residuals
qui reg iq `z1' `z2'
predict v2hat, residuals

qui reg lwage `z1' educ iq v1hat v2hat, vce(robust)
test v1hat v2hat

/*---------test result---------
. test v1hat v2hat

 (1)  v1hat = 0
 (2)  v2hat = 0
       F(  2,   711) =    4.10
            Prob > F =    0.0170
*/
\end{lstlisting}

    
    %problem boundary--------------------------------------------------------------------
    \item[6.4] Consider a structural linear model with unobserved variable $q$ :
    \[ y=\mathbf{x} \boldsymbol{\beta}+q+v, \quad \mathrm{E}(v \mid \mathbf{x}, q)=0 \]
    Suppose, in addition, that $\mathrm{E}(q \mid \mathbf{x})=\mathbf{x} \boldsymbol{\delta}$ for some $K \times 1$ vector $\boldsymbol{\delta} ;$ thus, $q$ and $\mathbf{x}$ are possibly correlated. (\textcolor{red}{see p137-140})
    \begin{enumerate}
        \item Show that $\mathrm{E}(y \mid \mathbf{x})$ is linear in $\mathbf{x}$. What consequences does this fact have for tests of functional form to detect the presence of $q$ ? Does it matter how strongly $q$ and $\mathbf{x}$ are correlated? Explain.
        
        \textbf{Answer:} From the Law of Iterated Expectation,
        \begin{align*}
            \E(v\mid\mathbf{x}) &= \E\left[ \E\left( v\mid\mathbf{x},q \right)\mid\mathbf{x} \right] = 0 \implies \\
            \E(y\mid\mathbf{x}) &= \mathbf{x} \boldsymbol{\beta} + \E(q\mid\mathbf{x}) + \E(v\mid\mathbf{x}) = \mathbf{x}(\bm{\beta}+\bm{\delta}) = \mathbf{x}\bm{\gamma}
        \end{align*}
        where $\bm{\gamma} = \bm{\beta}+\bm{\delta}$, so the $\E(v\mid\mathbf{x})$ is linear in $\mathbf{x}$. From the result above, there is no functional form misspecification in this conditional expectation so that no functional form test will detect the presence of $q$, no matter how strongly $q$ and $\mathbf{x}$ are correlated.
        
        \item Now add the assumptions $\operatorname{Var}(v \mid \mathbf{x}, q)=\sigma_{v}^{2}$ and $\operatorname{Var}(q \mid \mathbf{x})=\sigma_{q}^{2}$. Show that $\operatorname{Var}(y \mid \mathbf{x})$ is constant. (Hint: $\mathrm{E}(q v \mid \mathbf{x})=0$ by iterated expectations.) What does this fact imply about using tests for heteroskedasticity to detect omitted variables?
        
        \textbf{Answer:} Firstly, we will get the result of $\var(v\mid \mathbf{x})$: 
        \begin{align*}
            \E(v^2\mid \mathbf{x},q) &= \var(v\mid \mathbf{x},q) + \left[ \E(v\mid \mathbf{x},q) \right]^2 = \sigma^2_v \implies \\
            \E(v^2\mid \mathbf{x}) &= \E\left[ \E\left( v^2\mid \mathbf{x},q \right)\mid \mathbf{x} \right] = \sigma^2_v \implies \\
            \var(v\mid \mathbf{x}) &= \E(v^2\mid \mathbf{x}) - \left[ \E(v\mid \mathbf{x}) \right]^2 = \sigma^2_v
        \end{align*}
        Secondly, we will get the result of $\cov(q,v\mid \mathbf{x})$:
        \begin{align*}
            \E(qv\mid \mathbf{x}) &= \E\left[ \E\left( qv\mid \mathbf{x},q \right)\mid \mathbf{x} \right] = \E\left[ q \E \left( v\mid \mathbf{x},q \right)\mid \mathbf{x} \right] = 0 \implies \\
            \cov(q,v\mid \mathbf{x}) &= \E(qv\mid \mathbf{x}) - \E(q\mid \mathbf{x}) \E(v\mid \mathbf{x}) = 0
        \end{align*}
        Using the result above, we can obtain the result of $\var(y\mid \mathbf{x})$: 
        \[ \var(y\mid \mathbf{x}) = \var(q+v\mid \mathbf{x}) = \var(q\mid \mathbf{x}) + \var(v\mid \mathbf{x}) + 2\cov(q,v\mid \mathbf{x}) = \sigma^2_q + \sigma^2_v \]
        so that $y$ is conditionally homoskedastic. However, Along with $\E(y\mid \mathbf{x}) = \mathbf{x}\bm{\gamma}$ and $\var(y\mid \mathbf{x})$ being constant, a test for heteroskedasticity will always have a limiting $\chi^2$ distribution, so it will have no power for detecting omitted variables.
        
        \item Now write the equation as $y=\mathbf{x} \boldsymbol{\beta}+u,$ where $\mathrm{E}\left(\mathbf{x}^{\prime} u\right)=\mathbf{0}$ and $\operatorname{Var}(u \mid  \mathbf{x})=\sigma^{2}$. If $\mathrm{E}(u \mid  \mathbf{x}) \neq \mathrm{E}(u),$ argue that an $\mathrm{LM}$ test of the form (6.38) will detect ``heteroskedasticity'' in $u$, at least in large samples.
        
        \textbf{Answer:} Recall the equation \eqref{eq:6.4-c-1}: 
        \begin{gather}
            \text{regress } \hat{u}_i^2 \text{ on } 1, \mathbf{h}_i, \quad i = 1,2,\ldots,N \tag{6.38} \label{eq:6.4-c-1}
        \end{gather}
        If $\E(u\mid \mathbf{x}) = C$(a constant), then $\E(u) = \E(u\mid \mathbf{x}) = C$. Due to $\E(u) \neq \E(u\mid \mathbf{x})$, it implies that $\E(u\mid \mathbf{x})$ is not a constant and will generally be a function of $\mathbf{x}$ $\implies$ $\left[ \E(u\mid \mathbf{x}) \right]^2$ will generally be a function of $\mathbf{x}$ $\implies$ $\E(u^2\mid \mathbf{x}) = \var(u\mid \mathbf{x}) + \left[ \E(u\mid \mathbf{x}) \right]^2 = \sigma^2 + \left[ \E(u\mid \mathbf{x}) \right]^2$ will generally be a function of $\mathbf{x}$. Therefore, the LM test of the equation \eqref{eq:6.4-c-1} will be a significant result. That is, the LM test above will detect ``heteroskedasticity'' in $u$, at least in large samples, although there is not the case.
    \end{enumerate}
    
    \item[6.12] (Adapted) In the linear model $y = \mathbf{x}\bm{\beta} + u$, (\textcolor{red}{see p97, p130})
    \begin{enumerate}
        \item Under the appropriate homoskedasticity assumption, show that
        \[ \var\left( \hat{\bm{\beta}}_{\mathrm{2SLS}} - \hat{\bm{\beta}}_{\mathrm{OLS}} \right) = \var\left( \hat{\bm{\beta}}_{\mathrm{2SLS}} \right) - \var\left( \hat{\bm{\beta}}_{\mathrm{OLS}} \right) \]
        
        \textbf{Answer:} In the 2SLS and OLS, we know that
        \begin{gather*}
            \hat{\bm{\beta}}_{\mathrm{2SLS}} = \bm{\beta} + \left( \hat{\mathbf{X}}^\prime \hat{\mathbf{X}} \right)^{-1} \hat{\mathbf{X}}^\prime u \\
            \hat{\bm{\beta}}_{\mathrm{OLS}} = \bm{\beta} + \left( \mathbf{X}^\prime \mathbf{X} \right)^{-1} \mathbf{X}^\prime u
        \end{gather*}
        where $\hat{\mathbf{X}} = \mathbf{P}_{\mathbf{Z}} \mathbf{X}$ and $\mathbf{P}_{\mathbf{Z}} = \mathbf{Z}(\mathbf{Z}^\prime \mathbf{Z})^{-1} \mathbf{Z}^\prime$ is an idempotent and symmetric matrix. It is easy to show that $\hat{\mathbf{X}}^\prime \hat{\mathbf{X}} = \hat{\mathbf{X}}^\prime \mathbf{X}$, so
        \begin{align*}
            \cov\left( \hat{\bm{\beta}}_{\mathrm{2SLS}}, \hat{\bm{\beta}}_{\mathrm{OLS}} \right) &= \cov\left[ \left( \hat{\mathbf{X}}^\prime \hat{\mathbf{X}} \right)^{-1} \hat{\mathbf{X}}^\prime u, \left( \mathbf{X}^\prime \mathbf{X} \right)^{-1} \mathbf{X}^\prime u \right] \\
            &= \left( \hat{\mathbf{X}}^\prime \hat{\mathbf{X}} \right)^{-1} \hat{\mathbf{X}}^\prime \cov(u,u) \mathbf{X} \left( \mathbf{X}^\prime \mathbf{X} \right)^{-1} \\
            &= \sigma^2 \left( \hat{\mathbf{X}}^\prime \hat{\mathbf{X}} \right)^{-1} \hat{\mathbf{X}}^\prime \mathbf{X} \left( \mathbf{X}^\prime \mathbf{X} \right)^{-1} \\
            &= \sigma^2 \left( \mathbf{X}^\prime \mathbf{X} \right)^{-1} = \var\left( \hat{\bm{\beta}}_{\mathrm{OLS}} \right)
        \end{align*}
        After the same steps, 
        \[ \cov\left( \hat{\bm{\beta}}_{\mathrm{OLS}}, \hat{\bm{\beta}}_{\mathrm{2SLS}} \right) = \sigma^2 \left( \mathbf{X}^\prime \mathbf{X} \right)^{-1} = \var\left( \hat{\bm{\beta}}_{\mathrm{OLS}} \right) \]
        Using the results above, we can obtain
        \begin{align*}
            \var\left( \hat{\bm{\beta}}_{\mathrm{2SLS}} - \hat{\bm{\beta}}_{\mathrm{OLS}} \right) &= \var\left( \hat{\bm{\beta}}_{\mathrm{2SLS}} \right) + \var\left( \hat{\bm{\beta}}_{\mathrm{OLS}} \right) - \cov\left( \hat{\bm{\beta}}_{\mathrm{2SLS}}, \hat{\bm{\beta}}_{\mathrm{OLS}} \right) - \cov\left( \hat{\bm{\beta}}_{\mathrm{OLS}}, \hat{\bm{\beta}}_{\mathrm{2SLS}} \right) \\
            &= \var\left( \hat{\bm{\beta}}_{\mathrm{2SLS}} \right) - \var\left( \hat{\bm{\beta}}_{\mathrm{OLS}} \right)
        \end{align*}
        
        \item Prove that
        \[ \left( \hat{\bm{\beta}}_{\mathrm{2SLS}} - \hat{\bm{\beta}}_{\mathrm{OLS}} \right)^\prime \left[ \var\left( \hat{\bm{\beta}}_{\mathrm{2SLS}} \right) - \var\left( \hat{\bm{\beta}}_{\mathrm{OLS}} \right) \right]^{-1} \left( \hat{\bm{\beta}}_{\mathrm{2SLS}} - \hat{\bm{\beta}}_{\mathrm{OLS}} \right) \stackrel{a}{\scalebox{2}[1]{$\sim$}} \chi^2_k \]
        where $k$ is the number of parameters in the linear model.
        
        \textbf{Answer:} we know that
        \[ \sqrt{N}\left( \hat{\bm{\beta}}_{\mathrm{2SLS}} - \hat{\bm{\beta}}_{\mathrm{OLS}} \right) \stackrel{a}{\scalebox{2}[1]{$\sim$}} N \left( \mathbf{0}, \avar\left( \sqrt{N}\left( \hat{\bm{\beta}}_{\mathrm{2SLS}} - \hat{\bm{\beta}}_{\mathrm{OLS}} \right) \right) \right) \]
        and if $\mathbf{Y} \sim N_n(\mathbf{0},\bm{\Sigma})$, where $\bm{\Sigma}$ is positive definite,
        \[ \mathbf{Y}^\prime \bm{\Sigma}^{-1}\mathbf{Y} \sim \chi^2_n \]
        so, from the Continuous Mapping Theorem, we have
        \[ \left[ \sqrt{N}\left( \hat{\bm{\beta}}_{\mathrm{2SLS}} - \hat{\bm{\beta}}_{\mathrm{OLS}} \right) \right]^\prime \left[ \avar\left( \sqrt{N}\left( \hat{\bm{\beta}}_{\mathrm{2SLS}} - \hat{\bm{\beta}}_{\mathrm{OLS}} \right) \right) \right]^{-1} \left[ \sqrt{N}\left( \hat{\bm{\beta}}_{\mathrm{2SLS}} - \hat{\bm{\beta}}_{\mathrm{OLS}} \right) \right] \stackrel{a}{\scalebox{2}[1]{$\sim$}} \chi^2_k \]
        That is,
        \[ \left( \hat{\bm{\beta}}_{\mathrm{2SLS}} - \hat{\bm{\beta}}_{\mathrm{OLS}} \right)^\prime \left[ \var\left( \hat{\bm{\beta}}_{\mathrm{2SLS}} \right) - \var\left( \hat{\bm{\beta}}_{\mathrm{OLS}} \right) \right]^{-1} \left( \hat{\bm{\beta}}_{\mathrm{2SLS}} - \hat{\bm{\beta}}_{\mathrm{OLS}} \right) \stackrel{a}{\scalebox{2}[1]{$\sim$}} \chi^2_k \]
    \end{enumerate}
    
    \item[6.13] Referring to equations (6.17) and (6.18) , show that if $\mathrm{E}\left(u_{1} \mid \mathbf{z}\right)=0$ and $\mathrm{E}\left(u_{1} \mid \mathbf{z}, v_{2}\right)=\rho_{1} v_{2},$ then $\mathrm{E}\left(v_{2} \mid \mathbf{z}\right)=0$. (\textcolor{red}{see p128-129})
    
    \textbf{Answer:} From the Law of Iterated expectation,
    \[ 0 = \E(u_1 \mid \mathbf{z}) = \E\left[ \E(u_1 \mid \mathbf{z},v_2)\mid \mathbf{z} \right] = \rho_1 \E(v_2 \mid \mathbf{z}) \implies \E(v_2 \mid \mathbf{z}) = 0 \]
\end{enumerate}


\clearpage
\section*{Chapter 10}
\addcontentsline{toc}{section}{Chapter 10}

\begin{enumerate}
    %problem boundary--------------------------------------------------------------------
    \item[10.2] Suppose you have $T=2$ years of data on the same group of $N$ working individuals. Consider the following model of wage determination:
    \[ \log \left(wage_{it}\right)=\theta_{1}+\theta_{2} d 2_{t}+\mathbf{z}_{i t} \bm{\gamma}+\delta_{1} female_{i}+\delta_{2} d 2_{t} \cdot female_{i}+c_{i}+u_{i t} \]
    The unobserved effect $c_{i}$ is allowed to be correlated with $\mathbf{z}_{i t}$ and $female_{i}$. The variable $d 2_{t}$ is a time period indicator, where $d 2_{t}=1$ if $t=2$ and $d 2_{t}=0$ if $t=1$. In what follows, assume that
    \[ \mathrm{E}\left(u_{i t} \mid female_{i}, \mathbf{z}_{i 1}, \mathbf{z}_{i 2}, c_{i}\right)=0, \quad t=1,2 \]
    
    \begin{enumerate}
        \item Without further assumptions, what parameters in the log wage equation can be consistently estimated? (\textcolor{red}{see p300-304, p315-319})
        
        \textbf{Answer:} Assuming all elements of $\mathbf{z}_{it}$ are time-varying, the parameters $\theta_2, \delta_2 and \mathbf{\gamma}$ can be consistently estimated.
        
        \item Interpret the coefficients $\theta_{2}$ and $\delta_{2}$.
        
        \textbf{Answer:} Everything else equal, $\theta_2$ measures the growth in wage for men over the period and $\delta_2$ measures the difference in wage growth rate between women and men over the period.
        
        \item Write the log wage equation explicitly for the two time periods. Show that the differenced equation can be written as
        \[ \Delta \log \left(wage_{i}\right)=\theta_{2}+\Delta \mathbf{z}_{i} \bm{\gamma}+\delta_{2} female_{i}+\Delta u_{i} \]
        where $\Delta \log \left( wage_{i}\right)=\log \left(wage_{i 2}\right)-\log \left(wage_{i 1}\right)$, and so on.
        
        \textbf{Answer:} Write
        \begin{gather}
            \log \left(wage_{i1}\right)=\theta_{1}+\mathbf{z}_{i1} \bm{\gamma}+\delta_{1} female_{i}+c_{i}+u_{i1} \label{eq:10.2-1} \\
            \log \left(wage_{i2}\right)=\theta_{1}+\theta_{2}+\mathbf{z}_{i 2} \bm{\gamma}+\delta_{1} female_{i}+\delta_{2} \cdot female_{i}+c_{i}+u_{i2} \label{eq:10.2-2}
        \end{gather}
        Equations \eqref{eq:10.2-2} - \eqref{eq:10.2-1}, we have
        \begin{gather}
            \Delta \log \left(wage_{i}\right)=\theta_{2}+\Delta \mathbf{z}_{i} \bm{\gamma}+\delta_{2} female_{i}+\Delta u_{i} \label{eq:10.2-3}
        \end{gather}
        
        \item How would you test $\mathrm{H}_{0}: \delta_{2}=0$ if $\operatorname{Var}\left(\Delta u_{i} \mid \Delta \mathbf{z}_{i}, female_{i}\right)$ is not constant?
        
        \textbf{Answer:} If $\operatorname{Var}\left(\Delta u_{i} \mid \Delta \mathbf{z}_{i}, female_{i}\right)$ is not constant, we can test $\mathrm{H}_0: \delta_2 = 0$ with a robust variance matrix for equation \eqref{eq:10.2-3}:
        \[ \widehat{\avar\left(\hat{\boldsymbol{\beta}}_{F D}\right)}=\left(\Delta \mathbf{X}^{\prime} \Delta \mathbf{X}\right)^{-1}\left(\sum_{i=1}^{N} \Delta \mathbf{X}_{i}^{\prime} \widehat{\Delta u_i} \widehat{\Delta u_i}^{\prime} \Delta \mathbf{X}_{i}\right)\left(\Delta \mathbf{X}^{\prime} \Delta \mathbf{X}\right)^{-1} \]
    \end{enumerate}
    
    
    %problem boundary--------------------------------------------------------------------
    \item[10.3] For $T=2$ consider the standard unoberved effects model
    \[ y_{i t}=\mathbf{x}_{i t} \boldsymbol{\beta}+c_{i}+u_{i t}, \quad t=1,2 \]
    Let $\hat{\boldsymbol{\beta}}_{F E}$ and $\hat{\boldsymbol{\beta}}_{F D}$ denote the fixed effects and first difference estimators, respectively.
    
    \begin{enumerate}
        \item Show that the $\mathrm{FE}$ and $\mathrm{FD}$ estimates are numerically identical.
        
        \textbf{Answer:} Let $\Delta\mathbf{x}_i = \mathbf{x}_{i2} - \mathbf{x}_{i1}, \Delta y_i = y_{i2} - y_{i1}, \ddot{\mathbf{x}}_{it} = \mathbf{x}_{it} - \bar{\mathbf{x}}_i, \ddot{y}_{it} = y_{it} - \bar{y}_i$. It is easy to know that $\ddot{\mathbf{x}}_{i1} = -\Delta\mathbf{x}_i/2, \ddot{\mathbf{x}}_{i2} = \Delta\mathbf{x}_i/2$ and $\ddot{y}_{i1} = -\Delta y_i/2, \ddot{y}_{i2} = \Delta y_i/2$. Therefore, the fixed effects estimators can be written as
        \begin{align*}
            \hat{\boldsymbol{\beta}}_{F E}&=\left(\sum_{i=1}^{N} \sum_{t=1}^{T} \ddot{\mathbf{x}}_{i t}^{\prime} \ddot{\mathbf{x}}_{i t}\right)^{-1}\left(\sum_{i=1}^{N} \sum_{t=1}^{T} \ddot{\mathbf{x}}_{i t}^{\prime} \ddot{y}_{i t}\right) \\
            &= \left[\sum_{i=1}^{N}\left(\ddot{\mathbf{x}}_{i 1}^{\prime} \ddot{\mathbf{x}}_{i 1}+\ddot{\mathbf{x}}_{i 2}^{\prime} \ddot{\mathbf{x}}_{i 2}\right)\right]^{-1}\left[\sum_{i=1}^{N}\left(\ddot{\mathbf{x}}_{i 1}^{\prime} \ddot{y}_{i 1}+\ddot{\mathbf{x}}_{i 2}^{\prime} \ddot{y}_{i 2}\right)\right] \\
            &= \left(\sum_{i=1}^{N} \Delta \mathbf{x}_{i}^{\prime} \Delta \mathbf{x}_{i} / 2\right)^{-1}\left(\sum_{i=1}^{N} \Delta \mathbf{x}_{i}^{\prime} \Delta y_{i} / 2\right) \\
            &=\left(\sum_{i=1}^{N} \Delta \mathbf{x}_{i}^{\prime} \Delta \mathbf{x}_{i}\right)^{-1}\left(\sum_{i=1}^{N} \Delta \mathbf{x}_{i}^{\prime} \Delta y_{i}\right)=\hat{\boldsymbol{\beta}}_{FD}
        \end{align*}
        
        
        
        
        
        
        
        
        
        
        
        
        
        \item Show that the error variance estimates from the FE and FD methods are numerically identical.
    \end{enumerate}
    
    
    
    
    
    
    
    
    
    
    
    
    
    
    
    
    
    
    
    
    
    
    
    
    
    
    
\end{enumerate}


























\end{document}
