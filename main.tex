%xelatex 或 pdflatex 编译
%导言区
\documentclass[UTF8]{article} %UTF8编码
%preface for the main.tex
\usepackage[a4paper]{geometry} %设置纸张为A4大小
\usepackage{amsmath} %最常用的数学宏包
\allowdisplaybreaks %允许多行公式换页
\usepackage{amssymb} %提供了额外的数学字体和数学符号
\usepackage{mathtools} %提供了额外的数学工具,如dcases环境
\usepackage{lipsum} %英语假文宏包
\usepackage{graphicx} %插入图片宏包
\usepackage{float} %在浮动体中可使用H选项
\usepackage{extarrows} %提供了额外的箭头,如可随文字延长的各种箭头
\usepackage{array} %可增加更多的表格列限定符
\usepackage{dcolumn} %可使用小数点对齐的列
\usepackage{booktabs} %可使用三线表
\usepackage{bm} %可在数学环境中使用斜体加粗的命令
\usepackage[dvipsnames]{xcolor} %扩展版的color宏包

\usepackage{hyperref} %可使用超链接(含链接的跳转和一些命令)
\hypersetup{
	colorlinks=true,
	citecolor=magenta,%设置cite类命令超链接的颜色
	linkcolor=blue,%设置目录、脚注、ref等超链接的颜色
	urlcolor=violet,%设置网页超链接的颜色
}

\usepackage{listings}
\input{preface/listings-stata.tex} %导入Stata的listings配置文件
\lstset{ % Stata 语言全局设置
	language=Stata,
	basicstyle={\ttfamily},
% 	commentstyle={\color{gray}}, %设置注释的文字格式
	keywordstyle={[1]\color{Magenta}}, %设置首要关键字的文字格式
	keywordstyle={[2]\color{blue}}, %设置函数关键字的文字格式
	stringstyle={\color{Purple}}, %设置字符串的文字格式
	emph={}, %使前面 LaTeX 语言所设定的 emph 失效
	breaklines=true,
	tabsize=4,
	frame=trBL,
	frameround=fttt,
	flexiblecolumns=true,
}

\usepackage[amsmath,thmmarks]{ntheorem} %定理类环境宏包,如果前面使用amsmath宏包,则需加上amsmath宏包选项以避免出现未知问题,若需在定理环境末尾加上特定符号(如证毕符号),则需使用thmmarks宏包选项以使用\theoremsymbol{}命令。
{
	\theoremstyle{plain}
	\setlength{\theoremindent}{0em}
	\newtheorem{definition}{Definition}
}
{
	\theoremstyle{plain}
	\setlength{\theoremindent}{0em}
	\newtheorem{lemma}[definition]{Lemma}
}
{
	\theoremstyle{plain}
	\setlength{\theoremindent}{0em}
	\newtheorem{theorem}[definition]{Theorem}
}
{
	\theoremstyle{plain}
	\setlength{\theoremindent}{0em}
	\newtheorem{corollary}[definition]{Corollary}
}
{
	\theoremstyle{nonumberplain}
	\setlength{\theoremindent}{0em}
	\theorembodyfont{\normalfont}
	\theoremsymbol{$\blacksquare$} %在证明环境末尾加上一个证毕符号
	\newtheorem{proof}{Proof}
}

%定义新命令与新运算符
\DeclareMathOperator{\diff}{d\!}
\newcommand{\var}{\mathrm{Var}}
\newcommand{\avar}{\mathrm{Avar}}
\newcommand{\cov}{\mathrm{Cov}}
\newcommand{\E}{\mathrm{E}}

%重定义列表
\renewcommand{\labelenumii}{\theenumii.}

%设置段首行缩进
\setlength{\parindent}{0em}


%标题页设置
\title{
	Problem sets for Advanced Econometrics III
}
\author{Meiting Wang\thanks{Meiting Wang, Student ID: 202020111002, Email: wangmeiting92@gmail.com}}
\date{\today}




%正文区
\begin{document}

\maketitle
\tableofcontents
\pagenumbering{Roman}


\clearpage
\pagenumbering{arabic}
\section*{Chapter 5}
\addcontentsline{toc}{section}{Chapter 5}
\input{chapter5/chapter5.tex}


\clearpage
\section*{Chapter 6}
\addcontentsline{toc}{section}{Chapter 6}

\begin{enumerate}
    %problem boundary--------------------------------------------------------------------
    \item[6.2] In Problem 5.8b, test the null hypothesis that $educ$ and $IQ$ are exogenous in the equation estimated by 2SLS. (\textcolor{red}{see p126-127})
    
    \textbf{Answer:} Let $\mathbf{z}=(\mathbf{z}_1,\mathbf{z}_2)$, where $\mathbf{z}_1$ are the original exogenous variables and $\mathbf{z}_2$ are the IV variables. We first obtain the residuals($\hat{v}_1$ and $\hat{v}_2$) from the reduced form regressions below:
    \begin{align*}
        educ = \mathbf{z}\bm{\pi}_1 + v_1 \\ 
        IQ = \mathbf{z}\bm{\pi}_2 + v_2 \\ 
    \end{align*}
    Then, do the regression below and test whether the coefficients $\rho_1$ and $\rho_2$ are zero.
    \[ lwage = \mathbf{z}_1\bm{\delta}_1 + \alpha_1 educ + \alpha_2 IQ + \rho_1\hat{v}_1 + \rho_2\hat{v}_2 + error \]
    From the test result below, we find a strong evidence for the endogeneity of at least one of $educ$ and $IQ$. 
    \begin{lstlisting}
use nls80.dta, clear

local z1 "exper tenure married south urban black"
local z2 "kww meduc feduc sibs"

qui reg educ `z1' `z2'
predict v1hat, residuals
qui reg iq `z1' `z2'
predict v2hat, residuals

qui reg lwage `z1' educ iq v1hat v2hat, vce(robust)
test v1hat v2hat

/*---------test result---------
. test v1hat v2hat

 (1)  v1hat = 0
 (2)  v2hat = 0
       F(  2,   711) =    4.10
            Prob > F =    0.0170
*/
\end{lstlisting}

    
    %problem boundary--------------------------------------------------------------------
    \item[6.4] Consider a structural linear model with unobserved variable $q$ :
    \[ y=\mathbf{x} \boldsymbol{\beta}+q+v, \quad \mathrm{E}(v | \mathbf{x}, q)=0 \]
    Suppose, in addition, that $\mathrm{E}(q \mid \mathbf{x})=\mathbf{x} \boldsymbol{\delta}$ for some $K \times 1$ vector $\boldsymbol{\delta} ;$ thus, $q$ and $\mathbf{x}$ are possibly correlated. (\textcolor{red}{see p137-140})
    \begin{enumerate}
        \item Show that $\mathrm{E}(y | \mathbf{x})$ is linear in $\mathbf{x}$. What consequences does this fact have for tests of functional form to detect the presence of $q$ ? Does it matter how strongly $q$ and $\mathbf{x}$ are correlated? Explain.
        
        \textbf{Answer:} From the Law of Iterated Expectation,
        \begin{align*}
            \E(v|\mathbf{x}) &= \E\left[ \E\left( v|\mathbf{x},q \right)|\mathbf{x} \right] = 0 \implies \\
            \E(y|\mathbf{x}) &= \mathbf{x} \boldsymbol{\beta} + \E(q|\mathbf{x}) + \E(v|\mathbf{x}) = \mathbf{x}(\bm{\beta}+\bm{\delta}) = \mathbf{x}\bm{\gamma}
        \end{align*}
        where $\bm{\gamma} = \bm{\beta}+\bm{\delta}$, so the $\E(v|\mathbf{x})$ is linear in $\mathbf{x}$. From the result above, there is no functional form misspecification in this conditional expectation so that no functional form test will detect the presence of $q$, no matter how strongly $q$ and $\mathbf{x}$ are correlated.
        
        \item Now add the assumptions $\operatorname{Var}(v | \mathbf{x}, q)=\sigma_{v}^{2}$ and $\operatorname{Var}(q | \mathbf{x})=\sigma_{q}^{2}$. Show that $\operatorname{Var}(y | \mathbf{x})$ is constant. (Hint: $\mathrm{E}(q v | \mathbf{x})=0$ by iterated expectations.) What does this fact imply about using tests for heteroskedasticity to detect omitted variables?
        
        \textbf{Answer:} Firstly, we will get the result of $\var(v|\mathbf{x})$: 
        \begin{align*}
            \E(v^2|\mathbf{x},q) &= \var(v|\mathbf{x},q) + \left[ \E(v|\mathbf{x},q) \right]^2 = \sigma^2_v \implies \\
            \E(v^2|\mathbf{x}) &= \E\left[ \E\left( v^2|\mathbf{x},q \right)|\mathbf{x} \right] = \sigma^2_v \implies \\
            \var(v|\mathbf{x}) &= \E(v^2|\mathbf{x}) - \left[ \E(v|\mathbf{x}) \right]^2 = \sigma^2_v
        \end{align*}
        Secondly, we will get the result of $\cov(q,v|\mathbf{x})$:
        \begin{align*}
            \E(qv|\mathbf{x}) &= \E\left[ \E\left( qv|\mathbf{x},q \right)|\mathbf{x} \right] = \E\left[ q \E \left( v|\mathbf{x},q \right)|\mathbf{x} \right] = 0 \implies \\
            \cov(q,v|\mathbf{x}) &= \E(qv|\mathbf{x}) - \E(q|\mathbf{x}) \E(v|\mathbf{x}) = 0
        \end{align*}
        Using the result above, we can obtain the result of $\var(y|\mathbf{x})$: 
        \[ \var(y|\mathbf{x}) = \var(q+v|\mathbf{x}) = \var(q|\mathbf{x}) + \var(v|\mathbf{x}) + 2\cov(q,v|\mathbf{x}) = \sigma^2_q + \sigma^2_v \]
        so that $y$ is conditionally homoskedastic. However, Along with $\E(y|\mathbf{x}) = \mathbf{x}\bm{\gamma}$ and $\var(y|\mathbf{x})$ being constant, a test for heteroskedasticity will always have a limiting $\chi^2$ distribution, so it will have no power for detecting omitted variables.
        
        \item Now write the equation as $y=\mathbf{x} \boldsymbol{\beta}+u,$ where $\mathrm{E}\left(\mathbf{x}^{\prime} u\right)=\mathbf{0}$ and $\operatorname{Var}(u | \mathbf{x})=\sigma^{2}$. If $\mathrm{E}(u | \mathbf{x}) \neq \mathrm{E}(u),$ argue that an $\mathrm{LM}$ test of the form (6.38) will detect "heteroskedasticity" in $u,$ at least in large samples.
        
        \textbf{Answer:} Recall the equation \eqref{eq:6.4-c-1}: 
        \begin{gather}
            \text{regress } \hat{u}_i^2 \text{ on } 1, \mathbf{h}_i, \quad i = 1,2,\ldots,N \tag{6.38} \label{eq:6.4-c-1}
        \end{gather}
        If $\E(u|\mathbf{x}) = C$(a constant), then $\E(u) = \E(u|\mathbf{x}) = C$.
        
        
        
        
        
    \end{enumerate}
\end{enumerate}



\end{document}
