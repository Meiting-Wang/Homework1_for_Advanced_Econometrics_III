%xelatex 或 pdflatex 编译
%导言区
\documentclass[UTF8]{article} %UTF8编码
%preface for the main.tex
\usepackage[a4paper]{geometry} %设置纸张为A4大小
\usepackage{amsmath} %最常用的数学宏包
\allowdisplaybreaks %允许多行公式换页
\usepackage{amssymb} %提供了额外的数学字体和数学符号
\usepackage{mathtools} %提供了额外的数学工具,如dcases环境
\usepackage{lipsum} %英语假文宏包
\usepackage{graphicx} %插入图片宏包
\usepackage{float} %在浮动体中可使用H选项
\usepackage{extarrows} %提供了额外的箭头,如可随文字延长的各种箭头
\usepackage{array} %可增加更多的表格列限定符
\usepackage{dcolumn} %可使用小数点对齐的列
\usepackage{booktabs} %可使用三线表
\usepackage{bm} %可在数学环境中使用斜体加粗的命令
\usepackage[dvipsnames]{xcolor} %扩展版的color宏包

\usepackage{hyperref} %可使用超链接(含链接的跳转和一些命令)
\hypersetup{
	colorlinks=true,
	citecolor=magenta,%设置cite类命令超链接的颜色
	linkcolor=blue,%设置目录、脚注、ref等超链接的颜色
	urlcolor=violet,%设置网页超链接的颜色
}

\usepackage{listings}
\input{preface/listings-stata.tex} %导入Stata的listings配置文件
\lstset{ % Stata 语言全局设置
	language=Stata,
	basicstyle={\ttfamily},
% 	commentstyle={\color{gray}}, %设置注释的文字格式
	keywordstyle={[1]\color{Magenta}}, %设置首要关键字的文字格式
	keywordstyle={[2]\color{blue}}, %设置函数关键字的文字格式
	stringstyle={\color{Purple}}, %设置字符串的文字格式
	emph={}, %使前面 LaTeX 语言所设定的 emph 失效
	breaklines=true,
	tabsize=4,
	frame=trBL,
	frameround=fttt,
	flexiblecolumns=true,
}

\usepackage[amsmath,thmmarks]{ntheorem} %定理类环境宏包,如果前面使用amsmath宏包,则需加上amsmath宏包选项以避免出现未知问题,若需在定理环境末尾加上特定符号(如证毕符号),则需使用thmmarks宏包选项以使用\theoremsymbol{}命令。
{
	\theoremstyle{plain}
	\setlength{\theoremindent}{0em}
	\newtheorem{definition}{Definition}
}
{
	\theoremstyle{plain}
	\setlength{\theoremindent}{0em}
	\newtheorem{lemma}[definition]{Lemma}
}
{
	\theoremstyle{plain}
	\setlength{\theoremindent}{0em}
	\newtheorem{theorem}[definition]{Theorem}
}
{
	\theoremstyle{plain}
	\setlength{\theoremindent}{0em}
	\newtheorem{corollary}[definition]{Corollary}
}
{
	\theoremstyle{nonumberplain}
	\setlength{\theoremindent}{0em}
	\theorembodyfont{\normalfont}
	\theoremsymbol{$\blacksquare$} %在证明环境末尾加上一个证毕符号
	\newtheorem{proof}{Proof}
}

%定义新命令与新运算符
\DeclareMathOperator{\diff}{d\!}
\newcommand{\var}{\mathrm{Var}}
\newcommand{\avar}{\mathrm{Avar}}
\newcommand{\cov}{\mathrm{Cov}}
\newcommand{\E}{\mathrm{E}}

%重定义列表
\renewcommand{\labelenumii}{\theenumii.}

%设置段首行缩进
\setlength{\parindent}{0em}


%标题页设置
\title{
	Problem sets for Advanced Econometrics III
}
\author{Meiting Wang\thanks{Meiting Wang, Email: wangmeiting92@gmail.com}}
\date{\today}




%正文区
\begin{document}

\maketitle
\tableofcontents
\pagenumbering{Roman}


\clearpage
\pagenumbering{arabic}
\section*{Chapter 5}
\addcontentsline{toc}{section}{Chapter 5}
\input{chapter5/chapter5.tex}


\clearpage
\section*{Chapter 6}
\addcontentsline{toc}{section}{Chapter 6}
\begin{enumerate}
    %problem boundary--------------------------------------------------------------------
    \item[6.2] In Problem 5.8b, test the null hypothesis that $educ$ and $IQ$ are exogenous in the equation estimated by 2SLS. (\textcolor{red}{see p126-127})
    
    \textbf{Answer:} Let $\mathbf{z}=(\mathbf{z}_1,\mathbf{z}_2)$, where $\mathbf{z}_1$ are the original exogenous variables and $\mathbf{z}_2$ are the IV variables. We first obtain the residuals($\hat{v}_1$ and $\hat{v}_2$) from the reduced form regressions below:
    \begin{align*}
        educ = \mathbf{z}\bm{\pi}_1 + v_1 \\ 
        IQ = \mathbf{z}\bm{\pi}_2 + v_2 \\ 
    \end{align*}
    Then, do the regression below and test whether the coefficients $\rho_1$ and $\rho_2$ are zero.
    \[ lwage = \mathbf{z}_1\bm{\delta}_1 + \alpha_1 educ + \alpha_2 IQ + \rho_1\hat{v}_1 + \rho_2\hat{v}_2 + error \]
    From the test result below, we find a strong evidence for the endogeneity of at least one of $educ$ and $IQ$. 
    \begin{lstlisting}
use nls80.dta, clear

local z1 "exper tenure married south urban black"
local z2 "kww meduc feduc sibs"

qui reg educ `z1' `z2'
predict v1hat, residuals
qui reg iq `z1' `z2'
predict v2hat, residuals

qui reg lwage `z1' educ iq v1hat v2hat, vce(robust)
test v1hat v2hat

/*---------test result---------
. test v1hat v2hat

 (1)  v1hat = 0
 (2)  v2hat = 0
       F(  2,   711) =    4.10
            Prob > F =    0.0170
*/
\end{lstlisting}

    
    %problem boundary--------------------------------------------------------------------
    \item[6.4] Consider a structural linear model with unobserved variable $q$ :
    \[ y=\mathbf{x} \boldsymbol{\beta}+q+v, \quad \mathrm{E}(v \mid \mathbf{x}, q)=0 \]
    Suppose, in addition, that $\mathrm{E}(q \mid \mathbf{x})=\mathbf{x} \boldsymbol{\delta}$ for some $K \times 1$ vector $\boldsymbol{\delta} ;$ thus, $q$ and $\mathbf{x}$ are possibly correlated. (\textcolor{red}{see p137-140})
    \begin{enumerate}
        \item Show that $\mathrm{E}(y \mid \mathbf{x})$ is linear in $\mathbf{x}$. What consequences does this fact have for tests of functional form to detect the presence of $q$ ? Does it matter how strongly $q$ and $\mathbf{x}$ are correlated? Explain.
        
        \textbf{Answer:} From the Law of Iterated Expectation,
        \begin{align*}
            \E(v\mid\mathbf{x}) &= \E\left[ \E\left( v\mid\mathbf{x},q \right)\mid\mathbf{x} \right] = 0 \implies \\
            \E(y\mid\mathbf{x}) &= \mathbf{x} \boldsymbol{\beta} + \E(q\mid\mathbf{x}) + \E(v\mid\mathbf{x}) = \mathbf{x}(\bm{\beta}+\bm{\delta}) = \mathbf{x}\bm{\gamma}
        \end{align*}
        where $\bm{\gamma} = \bm{\beta}+\bm{\delta}$, so the $\E(v\mid\mathbf{x})$ is linear in $\mathbf{x}$. From the result above, there is no functional form misspecification in this conditional expectation so that no functional form test will detect the presence of $q$, no matter how strongly $q$ and $\mathbf{x}$ are correlated.
        
        \item Now add the assumptions $\operatorname{Var}(v \mid \mathbf{x}, q)=\sigma_{v}^{2}$ and $\operatorname{Var}(q \mid \mathbf{x})=\sigma_{q}^{2}$. Show that $\operatorname{Var}(y \mid \mathbf{x})$ is constant. (Hint: $\mathrm{E}(q v \mid \mathbf{x})=0$ by iterated expectations.) What does this fact imply about using tests for heteroskedasticity to detect omitted variables?
        
        \textbf{Answer:} Firstly, we will get the result of $\var(v\mid \mathbf{x})$: 
        \begin{align*}
            \E(v^2\mid \mathbf{x},q) &= \var(v\mid \mathbf{x},q) + \left[ \E(v\mid \mathbf{x},q) \right]^2 = \sigma^2_v \implies \\
            \E(v^2\mid \mathbf{x}) &= \E\left[ \E\left( v^2\mid \mathbf{x},q \right)\mid \mathbf{x} \right] = \sigma^2_v \implies \\
            \var(v\mid \mathbf{x}) &= \E(v^2\mid \mathbf{x}) - \left[ \E(v\mid \mathbf{x}) \right]^2 = \sigma^2_v
        \end{align*}
        Secondly, we will get the result of $\cov(q,v\mid \mathbf{x})$:
        \begin{align*}
            \E(qv\mid \mathbf{x}) &= \E\left[ \E\left( qv\mid \mathbf{x},q \right)\mid \mathbf{x} \right] = \E\left[ q \E \left( v\mid \mathbf{x},q \right)\mid \mathbf{x} \right] = 0 \implies \\
            \cov(q,v\mid \mathbf{x}) &= \E(qv\mid \mathbf{x}) - \E(q\mid \mathbf{x}) \E(v\mid \mathbf{x}) = 0
        \end{align*}
        Using the result above, we can obtain the result of $\var(y\mid \mathbf{x})$: 
        \[ \var(y\mid \mathbf{x}) = \var(q+v\mid \mathbf{x}) = \var(q\mid \mathbf{x}) + \var(v\mid \mathbf{x}) + 2\cov(q,v\mid \mathbf{x}) = \sigma^2_q + \sigma^2_v \]
        so that $y$ is conditionally homoskedastic. However, Along with $\E(y\mid \mathbf{x}) = \mathbf{x}\bm{\gamma}$ and $\var(y\mid \mathbf{x})$ being constant, a test for heteroskedasticity will always have a limiting $\chi^2$ distribution, so it will have no power for detecting omitted variables.
        
        \item Now write the equation as $y=\mathbf{x} \boldsymbol{\beta}+u,$ where $\mathrm{E}\left(\mathbf{x}^{\prime} u\right)=\mathbf{0}$ and $\operatorname{Var}(u \mid  \mathbf{x})=\sigma^{2}$. If $\mathrm{E}(u \mid  \mathbf{x}) \neq \mathrm{E}(u),$ argue that an $\mathrm{LM}$ test of the form (6.38) will detect ``heteroskedasticity'' in $u$, at least in large samples.
        
        \textbf{Answer:} Recall the equation \eqref{eq:6.4-c-1}: 
        \begin{gather}
            \text{regress } \hat{u}_i^2 \text{ on } 1, \mathbf{h}_i, \quad i = 1,2,\ldots,N \tag{6.38} \label{eq:6.4-c-1}
        \end{gather}
        If $\E(u\mid \mathbf{x}) = C$(a constant), then $\E(u) = \E(u\mid \mathbf{x}) = C$. Due to $\E(u) \neq \E(u\mid \mathbf{x})$, it implies that $\E(u\mid \mathbf{x})$ is not a constant and will generally be a function of $\mathbf{x}$ $\implies$ $\left[ \E(u\mid \mathbf{x}) \right]^2$ will generally be a function of $\mathbf{x}$ $\implies$ $\E(u^2\mid \mathbf{x}) = \var(u\mid \mathbf{x}) + \left[ \E(u\mid \mathbf{x}) \right]^2 = \sigma^2 + \left[ \E(u\mid \mathbf{x}) \right]^2$ will generally be a function of $\mathbf{x}$. Therefore, the LM test of the equation \eqref{eq:6.4-c-1} will be a significant result. That is, the LM test above will detect ``heteroskedasticity'' in $u$, at least in large samples, although there is not the case.
    \end{enumerate}
    
    \item[6.12] (Adapted) In the linear model $y = \mathbf{x}\bm{\beta} + u$, (\textcolor{red}{see p97, p130})
    \begin{enumerate}
        \item Under the appropriate homoskedasticity assumption, show that
        \[ \var\left( \hat{\bm{\beta}}_{\mathrm{2SLS}} - \hat{\bm{\beta}}_{\mathrm{OLS}} \right) = \var\left( \hat{\bm{\beta}}_{\mathrm{2SLS}} \right) - \var\left( \hat{\bm{\beta}}_{\mathrm{OLS}} \right) \]
        
        \textbf{Answer:} In the 2SLS and OLS, we know that
        \begin{gather*}
            \hat{\bm{\beta}}_{\mathrm{2SLS}} = \bm{\beta} + \left( \hat{\mathbf{X}}^\prime \hat{\mathbf{X}} \right)^{-1} \hat{\mathbf{X}}^\prime u \\
            \hat{\bm{\beta}}_{\mathrm{OLS}} = \bm{\beta} + \left( \mathbf{X}^\prime \mathbf{X} \right)^{-1} \mathbf{X}^\prime u
        \end{gather*}
        where $\hat{\mathbf{X}} = \mathbf{P}_{\mathbf{Z}} \mathbf{X}$ and $\mathbf{P}_{\mathbf{Z}} = \mathbf{Z}(\mathbf{Z}^\prime \mathbf{Z})^{-1} \mathbf{Z}^\prime$ is an idempotent and symmetric matrix. It is easy to show that $\hat{\mathbf{X}}^\prime \hat{\mathbf{X}} = \hat{\mathbf{X}}^\prime \mathbf{X}$, so
        \begin{align*}
            \cov\left( \hat{\bm{\beta}}_{\mathrm{2SLS}}, \hat{\bm{\beta}}_{\mathrm{OLS}} \right) &= \cov\left[ \left( \hat{\mathbf{X}}^\prime \hat{\mathbf{X}} \right)^{-1} \hat{\mathbf{X}}^\prime u, \left( \mathbf{X}^\prime \mathbf{X} \right)^{-1} \mathbf{X}^\prime u \right] \\
            &= \left( \hat{\mathbf{X}}^\prime \hat{\mathbf{X}} \right)^{-1} \hat{\mathbf{X}}^\prime \cov(u,u) \mathbf{X} \left( \mathbf{X}^\prime \mathbf{X} \right)^{-1} \\
            &= \sigma^2 \left( \hat{\mathbf{X}}^\prime \hat{\mathbf{X}} \right)^{-1} \hat{\mathbf{X}}^\prime \mathbf{X} \left( \mathbf{X}^\prime \mathbf{X} \right)^{-1} \\
            &= \sigma^2 \left( \mathbf{X}^\prime \mathbf{X} \right)^{-1} = \var\left( \hat{\bm{\beta}}_{\mathrm{OLS}} \right)
        \end{align*}
        After the same steps, 
        \[ \cov\left( \hat{\bm{\beta}}_{\mathrm{OLS}}, \hat{\bm{\beta}}_{\mathrm{2SLS}} \right) = \sigma^2 \left( \mathbf{X}^\prime \mathbf{X} \right)^{-1} = \var\left( \hat{\bm{\beta}}_{\mathrm{OLS}} \right) \]
        Using the results above, we can obtain
        \begin{align*}
            \var\left( \hat{\bm{\beta}}_{\mathrm{2SLS}} - \hat{\bm{\beta}}_{\mathrm{OLS}} \right) &= \var\left( \hat{\bm{\beta}}_{\mathrm{2SLS}} \right) + \var\left( \hat{\bm{\beta}}_{\mathrm{OLS}} \right) - \cov\left( \hat{\bm{\beta}}_{\mathrm{2SLS}}, \hat{\bm{\beta}}_{\mathrm{OLS}} \right) - \cov\left( \hat{\bm{\beta}}_{\mathrm{OLS}}, \hat{\bm{\beta}}_{\mathrm{2SLS}} \right) \\
            &= \var\left( \hat{\bm{\beta}}_{\mathrm{2SLS}} \right) - \var\left( \hat{\bm{\beta}}_{\mathrm{OLS}} \right)
        \end{align*}
        
        \item Prove that
        \[ \left( \hat{\bm{\beta}}_{\mathrm{2SLS}} - \hat{\bm{\beta}}_{\mathrm{OLS}} \right)^\prime \left[ \var\left( \hat{\bm{\beta}}_{\mathrm{2SLS}} \right) - \var\left( \hat{\bm{\beta}}_{\mathrm{OLS}} \right) \right]^{-1} \left( \hat{\bm{\beta}}_{\mathrm{2SLS}} - \hat{\bm{\beta}}_{\mathrm{OLS}} \right) \stackrel{a}{\scalebox{2}[1]{$\sim$}} \chi^2_k \]
        where $k$ is the number of parameters in the linear model.
        
        \textbf{Answer:} we know that
        \[ \sqrt{N}\left( \hat{\bm{\beta}}_{\mathrm{2SLS}} - \hat{\bm{\beta}}_{\mathrm{OLS}} \right) \stackrel{a}{\scalebox{2}[1]{$\sim$}} N \left( \mathbf{0}, \avar\left( \sqrt{N}\left( \hat{\bm{\beta}}_{\mathrm{2SLS}} - \hat{\bm{\beta}}_{\mathrm{OLS}} \right) \right) \right) \]
        and if $\mathbf{Y} \sim N_n(\mathbf{0},\bm{\Sigma})$, where $\bm{\Sigma}$ is positive definite,
        \[ \mathbf{Y}^\prime \bm{\Sigma}^{-1}\mathbf{Y} \sim \chi^2_n \]
        so, from the Continuous Mapping Theorem, we have
        \[ \left[ \sqrt{N}\left( \hat{\bm{\beta}}_{\mathrm{2SLS}} - \hat{\bm{\beta}}_{\mathrm{OLS}} \right) \right]^\prime \left[ \avar\left( \sqrt{N}\left( \hat{\bm{\beta}}_{\mathrm{2SLS}} - \hat{\bm{\beta}}_{\mathrm{OLS}} \right) \right) \right]^{-1} \left[ \sqrt{N}\left( \hat{\bm{\beta}}_{\mathrm{2SLS}} - \hat{\bm{\beta}}_{\mathrm{OLS}} \right) \right] \stackrel{a}{\scalebox{2}[1]{$\sim$}} \chi^2_k \]
        That is,
        \[ \left( \hat{\bm{\beta}}_{\mathrm{2SLS}} - \hat{\bm{\beta}}_{\mathrm{OLS}} \right)^\prime \left[ \var\left( \hat{\bm{\beta}}_{\mathrm{2SLS}} \right) - \var\left( \hat{\bm{\beta}}_{\mathrm{OLS}} \right) \right]^{-1} \left( \hat{\bm{\beta}}_{\mathrm{2SLS}} - \hat{\bm{\beta}}_{\mathrm{OLS}} \right) \stackrel{a}{\scalebox{2}[1]{$\sim$}} \chi^2_k \]
    \end{enumerate}
    
    \item[6.13] Referring to equations (6.17) and (6.18) , show that if $\mathrm{E}\left(u_{1} \mid \mathbf{z}\right)=0$ and $\mathrm{E}\left(u_{1} \mid \mathbf{z}, v_{2}\right)=\rho_{1} v_{2},$ then $\mathrm{E}\left(v_{2} \mid \mathbf{z}\right)=0$. (\textcolor{red}{see p128-129})
    
    \textbf{Answer:} From the Law of Iterated expectation,
    \[ 0 = \E(u_1 \mid \mathbf{z}) = \E\left[ \E(u_1 \mid \mathbf{z},v_2)\mid \mathbf{z} \right] = \rho_1 \E(v_2 \mid \mathbf{z}) \implies \E(v_2 \mid \mathbf{z}) = 0 \]
\end{enumerate}


\clearpage
\section*{Chapter 10}
\addcontentsline{toc}{section}{Chapter 10}

\begin{enumerate}
    %problem boundary--------------------------------------------------------------------
    \item[10.2] Suppose you have $T=2$ years of data on the same group of $N$ working individuals. Consider the following model of wage determination:
    \[ \log \left(wage_{it}\right)=\theta_{1}+\theta_{2} d 2_{t}+\mathbf{z}_{i t} \bm{\gamma}+\delta_{1} female_{i}+\delta_{2} d 2_{t} \cdot female_{i}+c_{i}+u_{i t} \]
    The unobserved effect $c_{i}$ is allowed to be correlated with $\mathbf{z}_{i t}$ and $female_{i}$. The variable $d 2_{t}$ is a time period indicator, where $d 2_{t}=1$ if $t=2$ and $d 2_{t}=0$ if $t=1$. In what follows, assume that
    \[ \mathrm{E}\left(u_{i t} \mid female_{i}, \mathbf{z}_{i 1}, \mathbf{z}_{i 2}, c_{i}\right)=0, \quad t=1,2 \]
    
    \begin{enumerate}
        \item Without further assumptions, what parameters in the log wage equation can be consistently estimated? (\textcolor{red}{see p300-304, p315-319})
        
        \textbf{Answer:} Assuming all elements of $\mathbf{z}_{it}$ are time-varying, the parameters $\theta_2, \delta_2 and \mathbf{\gamma}$ can be consistently estimated.
        
        \item Interpret the coefficients $\theta_{2}$ and $\delta_{2}$.
        
        \textbf{Answer:} Everything else equal, $\theta_2$ measures the growth in wage for men over the period and $\delta_2$ measures the difference in wage growth rate between women and men over the period.
        
        \item Write the log wage equation explicitly for the two time periods. Show that the differenced equation can be written as
        \[ \Delta \log \left(wage_{i}\right)=\theta_{2}+\Delta \mathbf{z}_{i} \bm{\gamma}+\delta_{2} female_{i}+\Delta u_{i} \]
        where $\Delta \log \left( wage_{i}\right)=\log \left(wage_{i 2}\right)-\log \left(wage_{i 1}\right)$, and so on.
        
        \textbf{Answer:} Write
        \begin{gather}
            \log \left(wage_{i1}\right)=\theta_{1}+\mathbf{z}_{i1} \bm{\gamma}+\delta_{1} female_{i}+c_{i}+u_{i1} \label{eq:10.2-1} \\
            \log \left(wage_{i2}\right)=\theta_{1}+\theta_{2}+\mathbf{z}_{i 2} \bm{\gamma}+\delta_{1} female_{i}+\delta_{2} \cdot female_{i}+c_{i}+u_{i2} \label{eq:10.2-2}
        \end{gather}
        Equations \eqref{eq:10.2-2} - \eqref{eq:10.2-1}, we have
        \begin{gather}
            \Delta \log \left(wage_{i}\right)=\theta_{2}+\Delta \mathbf{z}_{i} \bm{\gamma}+\delta_{2} female_{i}+\Delta u_{i} \label{eq:10.2-3}
        \end{gather}
        
        \item How would you test $\mathrm{H}_{0}: \delta_{2}=0$ if $\operatorname{Var}\left(\Delta u_{i} \mid \Delta \mathbf{z}_{i}, female_{i}\right)$ is not constant?
        
        \textbf{Answer:} If $\operatorname{Var}\left(\Delta u_{i} \mid \Delta \mathbf{z}_{i}, female_{i}\right)$ is not constant, we can test $\mathrm{H}_0: \delta_2 = 0$ with a robust variance matrix for equation \eqref{eq:10.2-3}:
        \[ \widehat{\avar\left(\hat{\boldsymbol{\beta}}_{F D}\right)}=\left(\Delta \mathbf{X}^{\prime} \Delta \mathbf{X}\right)^{-1}\left(\sum_{i=1}^{N} \Delta \mathbf{X}_{i}^{\prime} \widehat{\Delta u_i} \widehat{\Delta u_i}^{\prime} \Delta \mathbf{X}_{i}\right)\left(\Delta \mathbf{X}^{\prime} \Delta \mathbf{X}\right)^{-1} \]
    \end{enumerate}
    
    
    %problem boundary--------------------------------------------------------------------
    \item[10.3] For $T=2$ consider the standard unoberved effects model
    \[ y_{i t}=\mathbf{x}_{i t} \boldsymbol{\beta}+c_{i}+u_{i t}, \quad t=1,2 \]
    Let $\hat{\boldsymbol{\beta}}_{F E}$ and $\hat{\boldsymbol{\beta}}_{F D}$ denote the fixed effects and first difference estimators, respectively.
    
    \begin{enumerate}
        \item Show that the $\mathrm{FE}$ and $\mathrm{FD}$ estimates are numerically identical.
        
        \textbf{Answer:} Let $\Delta\mathbf{x}_i = \mathbf{x}_{i2} - \mathbf{x}_{i1}, \Delta y_i = y_{i2} - y_{i1}, \ddot{\mathbf{x}}_{it} = \mathbf{x}_{it} - \bar{\mathbf{x}}_i, \ddot{y}_{it} = y_{it} - \bar{y}_i$. It is easy to know that $\ddot{\mathbf{x}}_{i1} = -\Delta\mathbf{x}_i/2, \ddot{\mathbf{x}}_{i2} = \Delta\mathbf{x}_i/2$ and $\ddot{y}_{i1} = -\Delta y_i/2, \ddot{y}_{i2} = \Delta y_i/2$. Therefore, the fixed effects estimators can be written as
        \begin{align*}
            \hat{\boldsymbol{\beta}}_{F E}&=\left(\sum_{i=1}^{N} \sum_{t=1}^{T} \ddot{\mathbf{x}}_{i t}^{\prime} \ddot{\mathbf{x}}_{i t}\right)^{-1}\left(\sum_{i=1}^{N} \sum_{t=1}^{T} \ddot{\mathbf{x}}_{i t}^{\prime} \ddot{y}_{i t}\right) \\
            &= \left[\sum_{i=1}^{N}\left(\ddot{\mathbf{x}}_{i 1}^{\prime} \ddot{\mathbf{x}}_{i 1}+\ddot{\mathbf{x}}_{i 2}^{\prime} \ddot{\mathbf{x}}_{i 2}\right)\right]^{-1}\left[\sum_{i=1}^{N}\left(\ddot{\mathbf{x}}_{i 1}^{\prime} \ddot{y}_{i 1}+\ddot{\mathbf{x}}_{i 2}^{\prime} \ddot{y}_{i 2}\right)\right] \\
            &= \left(\sum_{i=1}^{N} \Delta \mathbf{x}_{i}^{\prime} \Delta \mathbf{x}_{i} / 2\right)^{-1}\left(\sum_{i=1}^{N} \Delta \mathbf{x}_{i}^{\prime} \Delta y_{i} / 2\right) \\
            &=\left(\sum_{i=1}^{N} \Delta \mathbf{x}_{i}^{\prime} \Delta \mathbf{x}_{i}\right)^{-1}\left(\sum_{i=1}^{N} \Delta \mathbf{x}_{i}^{\prime} \Delta y_{i}\right)=\hat{\boldsymbol{\beta}}_{FD}
        \end{align*}
        
        
        
        
        
        
        
        
        
        
        
        
        
        \item Show that the error variance estimates from the FE and FD methods are numerically identical.
    \end{enumerate}
    
    
    
    
    
    
    
    
    
    
    
    
    
    
    
    
    
    
    
    
    
    
    
    
    
    
    
\end{enumerate}


























\end{document}
