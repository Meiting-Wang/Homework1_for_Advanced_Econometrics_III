%preface for the main.tex
\usepackage[a4paper]{geometry} %设置纸张为A4大小
\usepackage{amsmath} %最常用的数学宏包
\allowdisplaybreaks %允许多行公式换页
\usepackage{amssymb} %提供了额外的数学字体和数学符号
\usepackage{mathtools} %提供了额外的数学工具,如dcases环境
\usepackage{lipsum} %英语假文宏包
\usepackage{graphicx} %插入图片宏包
\usepackage{float} %在浮动体中可使用H选项
\usepackage{extarrows} %提供了额外的箭头,如可随文字延长的各种箭头
\usepackage{array} %可增加更多的表格列限定符
\usepackage{dcolumn} %可使用小数点对齐的列
\usepackage{booktabs} %可使用三线表
\usepackage{bm} %可在数学环境中使用斜体加粗的命令
\usepackage[dvipsnames]{xcolor} %扩展版的color宏包

\usepackage{hyperref} %可使用超链接(含链接的跳转和一些命令)
\hypersetup{
	colorlinks=true,
	citecolor=magenta,%设置cite类命令超链接的颜色
	linkcolor=blue,%设置目录、脚注、ref等超链接的颜色
	urlcolor=violet,%设置网页超链接的颜色
}

\usepackage{listings}
% Description: Stata language definition for listings
% Author: WANG Meiting, PhD, Institute for Economic and Social Research, Jinan University
% Email: wangmeiting92@gmail.com
% Created on Sep 15, 2020
%
% Inspired by "https://git.io/JU8tn" and "https://git.io/JU8tG"

\lstdefinelanguage{Stata}{
    % 首要关键字
    morekeywords=[1]{qui,cls,eststo,ivregress,esttab,2sls,regress,reg,summarize,sum,display,di,generate,gen,bysort,use,import,delimited,predict,quietly,probit,margins,test,forvalues,foreach,set,sysuse,graph,local,global,export,twoway,replace,append,clear,if,else,dis,order,mat,matrix,mata,excel,cd,logit,aggregate,array,boolean,break,byte,case,catch,class,colvector,complex,const,continue,default,delegate,delete,do,double,eltypedef,end,enum,explicit,external,float,for,friend,function,goto,inline,int,long,namespace,new,numeric,NULL,operator,orgtypedef,pointer,polymorphic,pragma,private,protected,public,quad,real,return,rowvector,scalar,short,signed,static,strL,string,struct,super,switch,template,this,throw,transmorphic,try,typedef,typename,union,unsigned,using,vector,version,virtual,void,volatile,while,in,define,rown,rownames,coln,colnames,list,corr,ttest,tokenize},
    % 函数关键字
    morekeywords=[2]{bofd,Cdhms,Chms,Clock,clock,Cmdyhms,Cofc,cofC,Cofd,cofd,daily,date,day,dhms,dofb,dofC,dofc,dofh,dofm,dofq,dofw,dofy,dow,doy,halfyear,halfyearly,hh,hhC,hms,hofd,hours,mdy,mdyhms,minutes,mm,mmC,mofd,month,monthly,msofhours,msofminutes,msofseconds,qofd,quarter,quarterly,seconds,ss,ssC,tC,tc,td,th,tm,tq,tw,week,weekly,wofd,year,yearly,yh,ym,yofd,yq,yw,abs,ceil,cloglog,comb,digamma,exp,expm1,floor,int,invcloglog,invlogit,ln,ln1m,ln,ln1p,ln,lnfactorial,lngamma,log,log10,log1m,log1p,max,min,mod,reldif,round,sign,sqrt,trigamma,trunc,cholesky,coleqnumb,colnfreeparms,colnumb,colsof,det,diag,diag0cnt,el,get,hadamard,I,inv,invsym,issymmetric,J,matmissing,matuniform,mreldif,nullmat,roweqnumb,rownfreeparms,rownumb,rowsof,sweep,trace,vec,vecdiag,autocode,byteorder,c,_caller,chop,abs,clip,cond,e,fileexists,fileread,filereaderror,filewrite,float,fmtwidth,has_eprop,inlist,inrange,irecode,maxbyte,maxdouble,maxfloat,maxint,maxlong,mi,minbyte,mindouble,minfloat,minint,minlong,missing,r,recode,replay,return,s,scalar,smallestdouble,rbeta,rbinomial,rcauchy,rchi2,rexponential,rgamma,rhypergeometric,rigaussian,rlaplace,rlogistic,rnbinomial,rnormal,rpoisson,rt,runiform,runiformint,rweibull,rweibullph,tin,twithin,betaden,binomial,binomialp,binomialtail,binormal,cauchy,cauchyden,cauchytail,chi2,chi2den,chi2tail,dgammapda,dgammapdada,dgammapdadx,dgammapdx,dgammapdxdx,dunnettprob,exponential,exponentialden,exponentialtail,F,Fden,Ftail,gammaden,gammap,gammaptail,hypergeometric,hypergeometricp,ibeta,ibetatail,igaussian,igaussianden,igaussiantail,invbinomial,invbinomialtail,invcauchy,invcauchytail,invchi2,invchi2tail,invdunnettprob,invexponential,invexponentialtail,invF,invFtail,invgammap,invgammaptail,invibeta,invibetatail,invigaussian,invigaussiantail,invlaplace,invlaplacetail,invlogistic,invlogistictail,invnbinomial,invnbinomialtail,invnchi2,invnF,invnFtail,invnibeta,invnormal,invnt,invnttail,invpoisson,invpoissontail,invt,invttail,invtukeyprob,invweibull,invweibullph,invweibullphtail,invweibulltail,laplace,laplaceden,laplacetail,lncauchyden,lnigammaden,lnigaussianden,lniwishartden,lnlaplaceden,lnmvnormalden,lnnormal,lnnormalden,lnwishartden,logistic,logisticden,logistictail,nbetaden,nbinomial,nbinomialp,nbinomialtail,nchi2,nchi2den,nchi2tail,nF,nFden,nFtail,nibeta,normal,normalden,npnchi2,npnF,npnt,nt,ntden,nttail,poisson,poissonp,poissontail,t,tden,ttail,tukeyprob,weibull,weibullden,weibullph,weibullphden,weibullphtail,weibulltail,abbrev,char,collatorlocale,collatorversion,indexnot,plural,plural,real,regexm,regexr,regexs,soundex,soundex_nara,strcat,strdup,string,strofreal,string,strofreal,stritrim,strlen,strlower,strltrim,strmatch,strofreal,strofreal,strpos,strproper,strreverse,strrpos,strrtrim,strtoname,strtrim,strupper,subinstr,subinword,substr,tobytes,uchar,udstrlen,udsubstr,uisdigit,uisletter,ustrcompare,ustrcompareex,ustrfix,ustrfrom,ustrinvalidcnt,ustrleft,ustrlen,ustrlower,ustrltrim,ustrnormalize,ustrpos,ustrregexm,ustrregexra,ustrregexrf,ustrregexs,ustrreverse,ustrright,ustrrpos,ustrrtrim,ustrsortkey,ustrsortkeyex,ustrtitle,ustrto,ustrtohex,ustrtoname,ustrtrim,ustrunescape,ustrupper,ustrword,ustrwordcount,usubinstr,usubstr,word,wordbreaklocale,worcount,acos,acosh,asin,asinh,atan,atanh,cos,cosh,sin,sinh,tan,tanh,},
    % 注释
    morecomment=[l][\itshape\color{gray}]{//},
    morecomment=[s][\color{gray}]{/*}{*/},
    morecomment=[f][\itshape\color{gray}][0]{*}, %当*在首列时,为注释
    morecomment=[f][\itshape\color{gray}][1]{*}, %当*在第二列时,为注释
    morecomment=[f][\itshape\color{gray}][2]{*}, %当*在第三列时,为注释
    % 字符串
    morestring=[s]{`}{'},
    morestring=[s]{"}{"},
    morestring=[s]{"`}{'"},
    morestring=[s]{`"`}{'"'},
} %导入Stata的listings配置文件
\lstset{ % Stata 语言全局设置
	language=Stata,
	basicstyle={\ttfamily},
% 	commentstyle={\color{gray}}, %设置注释的文字格式
	keywordstyle={[1]\color{Magenta}}, %设置首要关键字的文字格式
	keywordstyle={[2]\color{blue}}, %设置函数关键字的文字格式
	stringstyle={\color{Purple}}, %设置字符串的文字格式
	emph={}, %使前面 LaTeX 语言所设定的 emph 失效
	breaklines=true,
	tabsize=4,
	frame=trBL,
	frameround=fttt,
	flexiblecolumns=true,
}

\usepackage[amsmath,thmmarks]{ntheorem} %定理类环境宏包,如果前面使用amsmath宏包,则需加上amsmath宏包选项以避免出现未知问题,若需在定理环境末尾加上特定符号(如证毕符号),则需使用thmmarks宏包选项以使用\theoremsymbol{}命令。
{
	\theoremstyle{plain}
	\setlength{\theoremindent}{0em}
	\newtheorem{definition}{Definition}
}
{
	\theoremstyle{plain}
	\setlength{\theoremindent}{0em}
	\newtheorem{lemma}[definition]{Lemma}
}
{
	\theoremstyle{plain}
	\setlength{\theoremindent}{0em}
	\newtheorem{theorem}[definition]{Theorem}
}
{
	\theoremstyle{plain}
	\setlength{\theoremindent}{0em}
	\newtheorem{corollary}[definition]{Corollary}
}
{
	\theoremstyle{nonumberplain}
	\setlength{\theoremindent}{0em}
	\theorembodyfont{\normalfont}
	\theoremsymbol{$\blacksquare$} %在证明环境末尾加上一个证毕符号
	\newtheorem{proof}{Proof}
}

%定义新命令与新运算符
\DeclareMathOperator{\diff}{d\!}
\newcommand{\var}{\mathrm{Var}}
\newcommand{\avar}{\mathrm{Avar}}
\newcommand{\cov}{\mathrm{Cov}}
\newcommand{\E}{\mathrm{E}}

%重定义列表
\renewcommand{\labelenumii}{\theenumii.}